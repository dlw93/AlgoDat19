\documentclass[aspectratio=169, t, envcountsect, smaller, notheorems]{beamer}
\usetheme[sectionpage=simple, titleformat=allsmallcaps]{metropolis}
\usepackage[ngerman]{babel}
\usepackage{amssymb, mathtools, cancel, pgfplots}
\usepackage[ruled, linesnumbered, vlined, ngerman, algosection]{algorithm2e}
\usepackage{hyperref}

\title[Algorithmen und Datenstrukturen]{Tutorium\\Algorithmen und Datenstrukturen}
\subtitle{Sommersemester 2019}
\author{David Luis Wiegandt}
\institute{Wissensmanagement in der Bioinformatik\\Institut für Informatik\\Humboldt-Universität zu Berlin}
\date{}

\setsansfont[Path = fonts/, UprightFont = *-Regular, BoldFont = *-Medium, ItalicFont = *-Italic]{FiraSans}

\resetcounteronoverlays{algocf}
\resetcounteronoverlays{theorem}

\uselanguage{german}
\languagepath{german}
\deftranslation[to=german]{Solution}{L\"osung}
\deftranslation[to=german]{Example}{Beispiel}
\deftranslation[to=german]{Task}{Aufgabe}
\deftranslation[to=german]{Remark}{Bemerkung}

\definecolor{hublue}{RGB}{0, 55, 108}
\definecolor{lighthublue}{RGB}{9, 80, 159}
\definecolor{tealblue}{RGB}{56, 134, 151}
\definecolor{madderlake}{RGB}{204, 41, 54}
\definecolor{ashgrey}{RGB}{174, 195, 176}
\definecolor{mandarin}{RGB}{239, 123, 69}

\newcommand{\teal}[1]{{\color{teal}#1}}
\newcommand{\red}[1]{{\color{madderlake}#1}}
\newcommand{\orange}[1]{{\color{mandarin}#1}}

\setbeamertemplate{frame footer}{\insertshorttitle}
\setbeamertemplate{itemize item}{\fontsize{3}{4}\raise2.5pt\hbox{\donotcoloroutermaths$\blacksquare$}}
\setbeamertemplate{itemize subitem}{\fontsize{2.5}{3.5}\raise2.5pt\hbox{\donotcoloroutermaths$\blacksquare$}}
\setbeamertemplate{itemize/enumerate body begin}{\normalsize}
\setbeamertemplate{itemize/enumerate subbody begin}{\small}
\setbeamertemplate{section in toc}[sections numbered]
\setbeamertemplate{theorems}[numbered]

\setbeamercolor{alerted text}{fg=lighthublue}
\setbeamercolor{block title example}{fg=hublue}
\setbeamercolor{footline}{fg=hublue}
\setbeamercolor{frametitle}{fg=hublue, bg=black!2}
\setbeamercolor{normal text}{fg=hublue}
\setbeamercolor{progress bar}{fg=hublue}

\setbeamerfont{alerted text}{shape=\itshape}
\setbeamerfont{author}{series=\bfseries}
\setbeamerfont{footline}{shape=\itshape}
\setbeamerfont{page number in head/foot}{size=\scriptsize}
\setbeamerfont{section title}{size=\LARGE}

% default styles for plots
\pgfplotsset{width=\textwidth, compat=1.9}
\pgfplotscreateplotcyclelist{hubcols}{
    {tealblue},
    {madderlake},
    {ashgrey}
}
\pgfplotsset{
    tikzDefaults/.style={
        cycle list name=hubcols,
        font=\small,
        line width = 1pt,
        legend style={
            at={(0.5,-0.2)},
            anchor=north, 
            column sep=7pt,
            draw=none,
            fill=black!2
        },
        legend columns=2,
        axis x line=bottom,
        axis y line=middle,
        tick style={draw=hublue, semithick},
        tick align=outside,
        every axis/.append style={font=\tiny},
    },
    samples=100
}

% remove rules from algos while retaining the ruled layout
\setlength{\algoheightrule}{0pt}
\setlength{\algotitleheightrule}{0pt}

% general math
\newcommand{\nat}{\mathbb{N}}
\newcommand{\natpos}{\mathbb{N}_{\geq 1}}
\newcommand{\real}{\mathbb{R}}
\newcommand{\realpos}{\mathbb{R}_{> 0}}
\newcommand{\powerset}{\mathcal{P}}
\newcommand{\bigO}{\mathcal{O}}
\DeclareMathOperator{\cupdot}{\mathbin{\mathaccent\cdot\cup}}
\DeclareMathOperator{\subsetfin}{\mathbin{\subset_{\mathsf{e}}}}
\newcommand{\defiff}{\mathrel{\vcentcolon\Leftrightarrow}}
\DeclarePairedDelimiter\norm{\lVert}{\rVert}

% query functions
\DeclareMathOperator{\adom}{adom}
\DeclareMathOperator{\vars}{var}
\DeclareMathOperator{\cons}{adom}
\DeclareMathOperator{\free}{frei}
\newcommand{\qfn}[1]{[\![#1]\!]}

% algorithms
\SetKwInOut{Input}{input}
\SetKwInOut{Output}{output}
\ResetInOut{output}
\SetKw{True}{true}
\SetKw{False}{false}

% theorem environments
\theoremstyle{definition}
\newtheorem{theorem}[algocf]{\translate{Theorem}}
\newtheorem{definition}[algocf]{\translate{Definition}}
\newtheorem{example}[algocf]{\translate{Example}}
\newtheorem{task}[algocf]{\translate{Task}}
\newtheorem*{solution}{\translate{Solution}}
\newtheorem*{remark}{Bemerkung}

% enumerate environments
\newenvironment{subtasks}{
    \setbeamertemplate{enumerate item}{(\alph{enumi})}
    \setbeamerfont{enumerate item}{series=\bfseries}
    \begin{enumerate}
}{
    \end{enumerate}
}

\begin{document}
    \maketitle
    
    \begin{frame}{Organisatorisches}
    \begin{block}{Zeit \& Ort}
        \begin{itemize}
            \item Mittwoch (RUD 25, 4.112) und Donnerstag (RUD 25, 4.113) von 13 bis 15 Uhr
            \item Teilnahme ist \alert{freiwillig} und \alert{flexibel}
            \begin{itemize}
                \item Ihr könnt stets an \alert{einem}, \alert{beiden} oder \alert{keinem} der Termine erscheinen
            \end{itemize}
        \end{itemize}
    \end{block}
    
    \begin{block}{Inhalte}
        \begin{itemize}
            \item Fokus ist die \alert{Wiederholung} von Inhalten aus Vorlesung und Übung
            \item Dar\"uber hinaus ist auch die \alert{Vertiefung} von Inhalten aus der Vorlesung m\"oglich
        \end{itemize}
    \end{block}
    
    \begin{block}{Bei Fragen}
        \begin{itemize}
            \item Im Tutorium oder jederzeit per Mail an \alert{\href{mailto:wigandtd@hu-berlin.de}{wigandtd@hu-berlin.de}}
        \end{itemize}
    \end{block}
    \end{frame}
    
    \begin{frame}{Behandelte Themen}
        \tableofcontents
    \end{frame}
    
    \section{Komplexit\"atsanalyse}

\begin{frame}{Hintergrund}
\begin{block}{Knappheit von Ressourcen}
    \begin{itemize}
        \item Speicherplatz (Register, RAM, HDD/SSD, Tape, ...)
        \item \alert{Rechenleistung} (Taktfrequenz)
    \end{itemize}
\end{block}

\begin{block}{Eine Problemstellung -- viele Lösungsansätze}
    \begin{itemize}
        \item Welchen Ansatz sollen wir nutzen?
        \item Ziel: Klassifiziere Lösungsansätze nach ihrer \alert{Eignung}
        \begin{itemize}
            \item Eignung $\approx$ Ressourcenbedarf
        \end{itemize}
    \end{itemize}
\end{block}
\end{frame}

\begin{frame}{Berechnungsmodell}
Wir analysieren \alert{abstrakte Algorithmen} -- folglich st\"utzen wir unsere Analysen auf \alert{abstrakte Maschinen}.

\medskip

\begin{columns}[T,onlytextwidth]
\begin{column}{0.54\textwidth}
Beispielsweise ist die \alert{Random Access Machine} eine zur \alert{Turingmaschine} \"aquivalente und f\"ur unsere Zwecke leichter zu handhabende Abstraktion einen realen Computers.
\end{column}
\begin{column}{0.4\textwidth}
\begin{tikzpicture}

\end{tikzpicture}
\end{column}
\end{columns}
\end{frame}

\begin{frame}<1-2>[label=first_example]{Ein einfaches Beispiel}
\begin{example}
Gegeben sei eine Liste natürlicher Zahlen $a_1, \dots, a_n \in \nat$.

Maximiere $x = a_i \cdot a_j$ mit $1 \leq i < j \leq n$.

\begin{quote}
    In Worten: Finde das \alert{gr\"o{\ss}te} Produkt, das sich aus zwei \alert{verschiedenen} Elementen der Liste bilden lässt.
\end{quote}

\begin{block}<2->{M\"ogliche Ans\"atze}
    \begin{enumerate}
        \item Betrachte alle möglichen Paare und berechne deren Produkt
        \pause
        \item<4-> Suche die beiden gr\"o{\ss}ten Elemente $a_i, a_j$ und gib $a_i \cdot a_j$ aus
    \end{enumerate}
\end{block}
\end{example}
\end{frame}

\begin{frame}
\frametitle<1>{Algorithmus}
\frametitle<2>{Algorithmus \& Analyse}
\begin{columns}[T,onlytextwidth]
    \begin{column}{0.42\textwidth}
        \vspace{-5pt}
        \begin{algorithm}[H]
        	\caption{Maximales Produkt 1}
        	\label{alg:max_prod}
        	\DontPrintSemicolon
        	\Input{$a_1, \dots, a_n \in \nat$}
            \Output{$\max \{ a_i \cdot a_j : 1 \leq i < j \leq n \}$}
        	$x \gets 0$\;
        	\For{$i \gets 1$ \KwTo $n$}{
        	    \For{$j \gets i + 1$ \KwTo $n$}{
        	        \If{$a_i \cdot a_j > x$}{
        	            $x \gets a_i \cdot a_j$\;
        	        }
                }
            }
            \Return{$x$}
        \end{algorithm}
    \end{column}
    \begin{column}<2>{0.52\textwidth}
        Die \"au{\ss}ere \textbf{for}-Schleife wird $n$-mal durchlaufen, die innere \textbf{for}-Schleife $n-1, n-2, \dots, 0$-mal
        \begin{itemize}
            \item Anwendung der \alert{gaußschen Summenformel} liefert $\frac{n \, (n-1)}{2}$
        \end{itemize}
        
        Der Aufwand des \textbf{if}-Statements ist von $n$ \alert{unabh\"angig}
        \begin{itemize}
            \item Multiplikationen, Vergleiche und Zuweisungen ben\"otigen lediglich \alert{konstant} viel Zeit
        \end{itemize}
        
        Insgesamt $\leq 4 \cdot \frac{n \, (n-1)}{2} + 2$ Instruktionen
    \end{column}
\end{columns}
\end{frame}

\againframe<3->{first_example}

\begin{frame}
\frametitle<1>{Algorithmus}
\frametitle<2>{Algorithmus \& Analyse}
\begin{columns}[T,onlytextwidth]
    \begin{column}{0.44\textwidth}
        \vspace{-5pt}
        \begin{algorithm}[H]
        	\caption{Maximales Produkt 2}
        	\label{alg:max_prod_ii}
        	\DontPrintSemicolon
        	\Input{$a_1, \dots, a_n \in \nat$}
            \Output{$\max \{ a_i \cdot a_j : 1 \leq i < j \leq n \}$}
        	$k \gets 0, \quad \ell \gets 0$\;
        	\For{$i \gets 1$ \KwTo $n$}{
        	    \If{$a_i > a_k$}{
        	        $\ell \gets k, \quad k \gets i$\;
    	        }
    	        \ElseIf{$a_i > a_{\ell}$}{
    	            $\ell \gets i$\;
    	        }
            }
            \Return{$a_k \cdot a_{\ell}$}
        \end{algorithm}
    \end{column}
    \begin{column}<2>{0.5\textwidth}
        Die \textbf{for}-Schleife wird genau $n$-mal durchlaufen
        
        \medskip
        
        Innerhalb der \textbf{if}-Statements f\"uhren wir lediglich grundlegende Operationen wie Vergleiche und Zuweisungen durch
        
        \medskip
        
        Insgesamt $\leq 3 \, n$ Instruktionen
    \end{column}
\end{columns}
\end{frame}

\begin{frame}{\MakeUppercase{$\bigO$}-Notation}
\begin{columns}[T,onlytextwidth]
\begin{column}{0.47\textwidth}
    \begin{definition}[$\bigO$-Notation]
    Seien $\teal{f}, \red{g}: \real \to \real$ Funktionen.
    Wir schreiben $\teal{f} \in \bigO(\red{g})$ genau dann, wenn es eine Konstante $\orange{c} \in \realpos$ gibt, sodass ab einem $n_0$ f\"ur alle $n \geq n_0$ gilt, dass $\teal{f}(n) \leq \orange{c} \cdot \red{g}(n)$.
    \end{definition}
    
    \medskip
    
    {\footnotesize Das hei{\ss}t, ab einem gewissen Punkt $n_0$ gibt es keinen \alert{wesentlichen} Unterschied mehr zwischen den Funktionswerten von $\teal{f}$ und $\red{g}$.}
\end{column}
\begin{column}{0.47\textwidth}
    \begin{tikzpicture}
        \begin{axis}[tikzDefaults, xmin=-4, xmax=12, ymin=0, ymax=500]
            \foreach \k in {1, ..., 8}{
                \addplot+[domain=-4:12, draw=mandarin!40, line width=0.5pt, forget plot]{1/2 * \k * x^2};
            }
            \addplot+[domain=-4:12, draw=teal]{2*x^2 - 7*x + 18};
            \addlegendentry{$2 \, x^2 - 7 \, x + 18$}
            \addplot+[domain=-4:12, draw=madderlake]{x^2};
            \addlegendentry{$x^2$}
        \end{axis}
    \end{tikzpicture}
\end{column}
\end{columns}
\end{frame}

\begin{frame}{Beispiel}
\begin{example}
Sei \teal{$f(n) \coloneqq 3 \, n + 5$} und \red{$g(n) \coloneqq 2 \, n$}.

Um zu zeigen, dass $\teal{f} \in \bigO(\red{g})$ ist, w\"ahlen wir \orange{$c=2$} und erhalten
\begin{equation*}
    \teal{\only<1>{3 \, n} \only<2->{\cancel{3 \, n}} + 5} \leq \orange{2} \cdot \red{2 \, n} = \only<1>{3 \, n} \only<2->{\cancel{3 \, n}} + n.
\end{equation*}
\only<2->{F\"ur $n \geq 5$ ist also $\teal{f(n)} \leq \orange{2} \cdot \red{g(n)}$ und somit $n_0 = 5$.}
\end{example}
\end{frame}

\begin{frame}{Bemerkungen}
\begin{remark}<+->
In der Praxis schreibt man h\"aufig $f(n) \in \bigO(g(n))$ anstelle von $f \in \bigO(g)$.

Beispielsweise ist $$\teal{2 \, n^2 + n} \in \bigO(\red{n^2})$$ \"aquivalent zu $$\teal{f} \in \bigO(\red{g}) \text{ f\"ur } \teal{f(n) \coloneqq 2 \, n^2 + n} \text{ und } \red{g(n) \coloneqq n^2}.$$
\end{remark}

\begin{remark}<+->
F\"ur Algorithmus \ref{alg:max_prod} konnten wir zeigen, dass dieser h\"ochstens $4 \cdot \frac{n \, (n-1)}{2} + 2 \in \bigO(n^2)$ Instruktionen ausf\"uhrt.
Demgegen\"uber ben\"otigt Algorithmus \ref{alg:max_prod_ii} nur h\"ochstens $3 \, n \in \bigO(n)$ Instruktionen und ist somit \alert{asymptotisch} effizienter ist als Algorithmus \ref{alg:max_prod}.
\end{remark}
\end{frame}

\begin{frame}{Eine leichte Aufgabe}
\begin{task}<1->
Sei $f(n) \coloneqq 6 \, n + 7$.

Zeigen Sie, dass $f(n) \in \bigO(n)$ ist, d.h. finden Sie $c \in \realpos, n_0 \in \real$, sodass f\"ur alle $n \geq n_0$ gilt, dass $f(n) \leq c \cdot n$.
\end{task}

\begin{solution}<2->
F\"ur $c = 7$ ist $6 \, n + 7 \leq 7 \, n = 6 \, n + n$ f\"ur alle $n \geq n_0 \coloneqq 7$.
\end{solution}
\end{frame}

\begin{frame}{Mehr Aufgaben}
\begin{task}
Beweisen oder widerlegen Sie:
\begin{subtasks}
    \item $12 \, n - 4 \in \bigO(n)$
    \item $7 \, n + 1 \in \bigO(n^2)$
    \item $\log(n^2) \in \bigO(\log n)$
\end{subtasks}
\end{task}
\end{frame}

\begin{frame}{Alternative Definition}
\begin{definition}[$\bigO$-Notation]
Seien $\teal{f}, \red{g}: \real \to \real$ Funktionen.
Wir schreiben $\teal{f} \in \bigO(\red{g})$ genau dann, wenn $\limsup_{n \to \infty} \frac{\teal{f(n)}}{\red{g(n)}} < \infty$ und $\red{g(n)} > 0$ f\"ur alle $n \in \nat$.
\end{definition}

Das folgende Theorem liefert ein bei dieser Charakterisierung h\"aufig sehr n\"utzliches Werkzeug.

\begin{theorem}[Satz von L'Hospital]
...
\end{theorem}
\end{frame}

\begin{frame}{Ein Beispiel zum Satz von L'Hospital}
\begin{example}

\end{example}
\end{frame}

\begin{frame}{Aufgaben zum Satz von L'Hospital}
\begin{task}

\end{task}
\end{frame}

\begin{frame}{Aufgabe zur Analyse von Algorithmen}
\begin{task}

\end{task}
\end{frame}

\end{document}
