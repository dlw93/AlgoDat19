\documentclass[aspectratio=169, t, envcountsect, smaller, notheorems]{beamer}
\usetheme[sectionpage=simple, titleformat=allsmallcaps]{metropolis}
\usepackage[ngerman]{babel}
\usepackage{amssymb, mathtools, cancel, pgfplots, nicefrac, stmaryrd}
\usepackage[ruled, linesnumbered, vlined, ngerman, algosection]{algorithm2e}
\usepackage{array, csquotes, hyperref, tasks, multicol}

\MakeOuterQuote{"}

\newcolumntype{L}[1]{>{\raggedleft\arraybackslash}p{#1}}

\usetikzlibrary{automata, positioning, arrows, patterns, fillbetween}

\title[Algorithmen und Datenstrukturen]{Algorithmen und Datenstrukturen}
\subtitle{Sommersemester 2019}
\author{David Luis Wiegandt}
\institute{Wissensmanagement in der Bioinformatik\\Institut für Informatik\\Humboldt-Universität zu Berlin}
\date{}
\titlegraphic{\includegraphics[width=70pt, height=70pt]{images/husiegel_bw.eps}}

\setsansfont[Path = fonts/, UprightFont = *-Regular, BoldFont = *-Medium, ItalicFont = *-Italic]{FiraSans}

\resetcounteronoverlays{algocf}
\resetcounteronoverlays{theorem}

\settasks{label-format=\bfseries, counter-format=(tsk[a]), label-offset=.6em}

\uselanguage{german}
\languagepath{german}
\deftranslation[to=german]{Solution}{L\"osung}
\deftranslation[to=german]{Example}{Beispiel}
\deftranslation[to=german]{Task}{Aufgabe}
\deftranslation[to=german]{Remark}{Bemerkung}
\deftranslation[to=german]{Corollary}{Korollar}

\definecolor{hublue}{RGB}{0, 55, 108}
\definecolor{lighthublue}{RGB}{9, 80, 159}
\definecolor{tealblue}{RGB}{56, 134, 151}
\definecolor{madderlake}{RGB}{204, 41, 54}
\definecolor{ashgrey}{RGB}{174, 195, 176}
\definecolor{mandarin}{RGB}{239, 123, 69}
\definecolor{amaranth}{RGB}{160, 26, 125}

\newcommand<>{\emphlightblue}[1]{\textit#2{\color#2{lighthublue}#1}}
\newcommand<>{\emphteal}[1]{\textit#2{\color#2{teal}#1}}
\newcommand<>{\emphred}[1]{\textit#2{\color#2{madderlake}#1}}
\newcommand<>{\emphorange}[1]{\textit#2{\color#2{mandarin}#1}}
\newcommand<>{\emphpurple}[1]{\textit#2{\color#2{amaranth}#1}}

\newcommand<>{\lightblue}[1]{{\color#2{lighthublue}#1}}
\newcommand<>{\teal}[1]{{\color#2{teal}#1}}
\newcommand<>{\red}[1]{{\color#2{madderlake}#1}}
\newcommand<>{\orange}[1]{{\color#2{mandarin}#1}}
\newcommand<>{\purple}[1]{{\color#2{amaranth}#1}}

\makeatletter
\patchcmd{\beamer@sectionintoc}{\vfill}{\vskip\itemsep}{}{}

\setbeamertemplate{frame footer}{\insertshorttitle}
\setbeamertemplate{itemize item}{\fontsize{3}{4}\raise2.5pt\hbox{\donotcoloroutermaths$\blacksquare$}}
\setbeamertemplate{itemize subitem}{\fontsize{2.5}{3.5}\raise2.5pt\hbox{\donotcoloroutermaths$\blacksquare$}}
\setbeamertemplate{itemize/enumerate body begin}{\normalsize}
\setbeamertemplate{itemize/enumerate subbody begin}{\small}
\setbeamertemplate{section in toc}[sections numbered]
\setbeamertemplate{subsection in toc}[subsections numbered]
\setbeamertemplate{theorems}[numbered]

\setbeamercolor{alerted text}{fg=lighthublue}
\setbeamercolor{block title alerted}{fg=madderlake}
\setbeamercolor{footline}{fg=hublue}
\setbeamercolor{frametitle}{fg=hublue, bg=black!2}
\setbeamercolor{normal text}{fg=hublue}
\setbeamercolor{progress bar}{fg=hublue}
\setbeamercolor{title page header}{fg=white, bg=hublue}

\setbeamerfont{alerted text}{shape=\itshape}
\setbeamerfont{author}{series=\bfseries, size=\small}
\setbeamerfont{footline}{shape=\itshape}
\setbeamerfont{page number in head/foot}{size=\scriptsize}
\setbeamerfont{section title}{size=\LARGE}
\setbeamerfont{title}{size=\LARGE}

\setbeamertemplate{title page}{
\vspace*{-1pt}
\begin{beamercolorbox}[wd=\paperwidth,ht=0.618\paperheight]{title page header}
    \vfill
    \par
    \raggedright%
    \leftskip28pt%
    {\usebeamerfont{title} tutorium \\ \vspace{5pt} \inserttitle}
    \par
    \vspace*{3pt}
    {\usebeamerfont{subtitle}\insertsubtitle}
    \vspace*{-5pt}
    \vfill
    \tikz[remember picture, overlay] \node[anchor=center] at (0.8\paperwidth,0) {\inserttitlegraphic};
\end{beamercolorbox}
\vspace*{2pt}
{\usebeamerfont{author}\beamer@shortauthor}

\medskip
\fontsize{10}{8}\selectfont
\par

{\usebeamerfont{institute}\insertinstitute}
\vfill
}
\makeatother

% default styles for plots
\pgfplotsset{width=\textwidth, compat=1.9}
\pgfplotscreateplotcyclelist{hubcols}{
    {tealblue},
    {madderlake},
    {ashgrey}
}
\pgfplotsset{
    tikzDefaults/.style={
        cycle list name=hubcols,
        font=\small,
        line width = 1pt,
        legend style={
            at={(0.5,-0.2)},
            anchor=north, 
            column sep=7pt,
            draw=none,
            fill=black!2
        },
        legend columns=2,
        axis x line=bottom,
        axis y line=middle,
        tick style={draw=hublue, semithick},
        tick align=center
        every axis/.append style={font=\tiny},
    },
    samples=100
}

\tikzset{
    node distance=2cm,
    every state/.style={
        thick,
        fill=lighthublue!7
    },
    initial text={},
    double distance=2pt,
    every edge/.style={
        draw,
        ->,
        >=stealth,
        auto,
        thick,
        text=hublue
    }
}

% remove rules from algos while retaining the ruled layout
\setlength{\algoheightrule}{0pt}
\setlength{\algotitleheightrule}{0pt}

\let\leq=\leqslant
\let\geq=\geqslant

% general math
\newcommand{\nat}{\mathbb{N}}
\newcommand{\natpos}{\mathbb{N}_{\geq 1}}
\newcommand{\real}{\mathbb{R}}
\newcommand{\realpos}{\mathbb{R}_{> 0}}
\newcommand{\powerset}{\mathcal{P}}
\newcommand{\bigO}{\mathcal{O}}
\DeclareMathOperator{\cupdot}{\mathbin{\mathaccent\cdot\cup}}
\DeclareMathOperator{\subsetfin}{\mathbin{\subset_{\mathsf{e}}}}
\newcommand{\defiff}{\mathrel{\vcentcolon\Leftrightarrow}}
\DeclarePairedDelimiter\norm{\lVert}{\rVert}

% query functions
\DeclareMathOperator{\adom}{adom}
\DeclareMathOperator{\vars}{var}
\DeclareMathOperator{\cons}{adom}
\DeclareMathOperator{\free}{frei}
\newcommand{\qfn}[1]{[\![#1]\!]}

% algorithms
\SetKwInOut{Input}{input}
\SetKwInOut{Output}{output}
\ResetInOut{output}
\SetKw{True}{true}
\SetKw{False}{false}

% theorem environments
\theoremstyle{definition}
\newtheorem{theorem}[algocf]{\translate{Theorem}}
\newtheorem{definition}[algocf]{\translate{Definition}}
\newtheorem{example}[algocf]{\translate{Example}}
\newtheorem{task}[algocf]{\translate{Task}}
\newtheorem*{solution}{\translate{Solution}}
\newtheorem*{remark}{Bemerkung}

% enumerate environments
\newenvironment{subtasks}{
    \setbeamertemplate{enumerate item}{(\alph{enumi})}
    \setbeamerfont{enumerate item}{series=\bfseries}
    \begin{enumerate}
}{
    \end{enumerate}
}

\begin{document}
    \maketitle
    
    \begin{frame}{Organisatorisches}
    \begin{block}{Zeit \& Ort}
        \begin{itemize}
            \item Mittwoch (RUD 25, 4.112) und Donnerstag (RUD 25, 4.113) von 13 bis 15 Uhr
            \item Teilnahme ist \alert{freiwillig} und \alert{flexibel}
            \begin{itemize}
                \item Ihr könnt stets an \alert{einem}, \alert{beiden} oder \alert{keinem} der Termine erscheinen
            \end{itemize}
        \end{itemize}
    \end{block}
    
    \begin{block}{Inhalte}
        \begin{itemize}
            \item Fokus ist die \alert{Wiederholung} von Inhalten aus Vorlesung und Übung
            \item Dar\"uber hinaus ist auch die \alert{Vertiefung} von Inhalten aus der Vorlesung m\"oglich
        \end{itemize}
    \end{block}
    
    \begin{block}{Bei Fragen}
        \begin{itemize}
            \item Im Tutorium oder per Mail an \alert{\href{mailto:wigandtd@hu-berlin.de}{wigandtd@hu-berlin.de}}
        \end{itemize}
    \end{block}
    \end{frame}
    
    \begin{frame}{Behandelte Themen}
        \tableofcontents
    \end{frame}
    
    \section{Laufzeitanalyse}

\begin{frame}{Hintergrund}
\begin{block}{Knappheit von Ressourcen}
    \begin{itemize}
        \item Speicherplatz (Register, RAM, HDD/SSD, Tape, ...)
        \item \alert{Rechenleistung} (Taktfrequenz)
    \end{itemize}
\end{block}
\begin{block}{Eine Problemstellung -- viele Lösungsansätze}
    \begin{itemize}
        \item Welchen Ansatz sollen wir nutzen?
        \item Ziel: Klassifiziere Lösungsansätze nach ihrer \alert{Eignung}
        \begin{itemize}
            \item Eignung $\approx$ Ressourcenbedarf
        \end{itemize}
    \end{itemize}
\end{block}
\begin{block}{Berechnungsmodell}
    \begin{itemize}
        \item Wir analysieren \alert{abstrakte Algorithmen} -- folglich st\"utzen wir unsere Analysen auf \alert{abstrakte Maschinen}
    \end{itemize}
\end{block}
\end{frame}

\begin{frame}<1-2>[label=first_example]{Ein einfaches Beispiel}
\begin{example}
Gegeben sei eine Liste natürlicher Zahlen $a_1, \dots, a_n \in \nat$.

Maximiere $x = a_i \cdot a_j$ mit $1 \leq i < j \leq n$.

\begin{quote}
    In Worten: Finde das \alert{gr\"o{\ss}te} Produkt, das sich aus zwei \alert{verschiedenen} Elementen der Liste bilden lässt.
\end{quote}

\begin{block}<2->{M\"ogliche Ans\"atze}
    \begin{enumerate}
        \item Betrachte alle möglichen Paare und berechne deren Produkt
        \pause
        \item<4-> Suche die beiden gr\"o{\ss}ten Elemente $a_i, a_j$ und gib $a_i \cdot a_j$ aus
    \end{enumerate}
\end{block}
\end{example}
\end{frame}

\begin{frame}
\frametitle<1>{Algorithmus}
\frametitle<2>{Algorithmus \& Analyse}
\begin{columns}[T,onlytextwidth]
    \begin{column}{0.43\textwidth}
        \vspace{-5pt}
        \begin{algorithm}[H]
        	\caption{Maximales Produkt 1}
        	\label{alg:max_prod}
        	\DontPrintSemicolon
        	\Input{$a_1, \dots, a_n \in \nat$}
            \Output{$\max \{ a_i \cdot a_j : 1 \leq i < j \leq n \}$}
        	$x \gets 0$\;
        	\For{$i \gets 1$ \KwTo $n - 1$}{
        	    \For{$j \gets i + 1$ \KwTo $n$}{
        	        \If{$a_i \cdot a_j > x$}{
        	            $x \gets a_i \cdot a_j$\;
        	        }
                }
            }
            \Return{$x$}
        \end{algorithm}
    \end{column}
    \begin{column}<2>{0.51\textwidth}
        Die \"au{\ss}ere \textbf{for}-Schleife wird $(n-1)$-mal durchlaufen, die innere \textbf{for}-Schleife $n-1, n-2, \dots, 1$-mal
        \begin{itemize}
            \item Anwendung der \alert{gaußschen Summenformel} liefert $\frac{n \, (n-1)}{2}$
        \end{itemize}
        
        Der Aufwand des \textbf{if}-Statements ist von $n$ \alert{unabh\"angig}
        \begin{itemize}
            \item Multiplikationen, Vergleiche und Zuweisungen ben\"otigen lediglich \alert{konstant} viel Zeit
        \end{itemize}
        
        Insgesamt $\leq 4 \cdot \frac{n \, (n-1)}{2} + 2$ Instruktionen
    \end{column}
\end{columns}
\end{frame}

\againframe<3->{first_example}

\begin{frame}
\frametitle<1>{Algorithmus}
\frametitle<2>{Algorithmus \& Analyse}
\begin{columns}[T,onlytextwidth]
    \begin{column}{0.44\textwidth}
        \vspace{-5pt}
        \begin{algorithm}[H]
        	\caption{Maximales Produkt 2}
        	\label{alg:max_prod_ii}
        	\DontPrintSemicolon
        	\Input{$a_1, \dots, a_n \in \nat$}
            \Output{$\max \{ a_i \cdot a_j : 1 \leq i < j \leq n \}$}
        	$k \gets 0, \quad \ell \gets 0$\;
        	\For{$i \gets 1$ \KwTo $n$}{
        	    \If{$a_i > a_k$}{
        	        $\ell \gets k, \quad k \gets i$\;
    	        }
    	        \ElseIf{$a_i > a_{\ell}$}{
    	            $\ell \gets i$\;
    	        }
            }
            \Return{$a_k \cdot a_{\ell}$}
        \end{algorithm}
    \end{column}
    \begin{column}<2>{0.5\textwidth}
        Die \textbf{for}-Schleife wird genau $n$-mal durchlaufen
        
        \medskip
        
        Innerhalb der \textbf{if}-Statements f\"uhren wir lediglich grundlegende Operationen wie Vergleiche und Zuweisungen durch
        
        \medskip
        
        Insgesamt $\leq 3 \, n + 4$ Instruktionen
    \end{column}
\end{columns}
\end{frame}

\begin{frame}<1-2>[label=second_example]{Ein komplexeres Beispiel}
\begin{example}
Wir nennen eine Sequenz $a_1, \dots, a_n \in \nat$ \alert{partitionierbar}, falls  es $k, \ell \in \{1, \dots, n\}$ gibt, sodass $\sum_{i=k}^{\ell} a_i = \frac{1}{2} \cdot  \sum_{i=1}^{n} a_i$.

Entscheide f\"ur eine Sequenz $a_1, \dots, a_n$ mit $\{ a_1, \dots, a_n \} = \{1, \dots, n\}$, ob diese partitionierbar ist.

\begin{quote}
    In Worten: Entscheide f\"ur eine \alert{Permutation} der nat\"urlichen Zahlen von $1$ bis $n$, ob es eine \alert{zusammenh\"angende Subsequenz} gibt, deren Summe die \alert{H\"alfte} der Summe der Gesamtsequenz ist.
\end{quote}

\begin{block}<2->{M\"ogliche Ans\"atze}
    \begin{enumerate}
        \item Berechne die Summe aller möglichen Subsequenzen \emph{from scratch}
        \pause
        \item<4-> Nutze die Distributivit\"at der Addition, um redundante Berechnungen zu vermeiden
    \end{enumerate}
\end{block}
\end{example}
\end{frame}

\begin{frame}
\frametitle<1->{Algorithmus}
\frametitle<2->{Algorithmus \& Analyse}
\begin{columns}[T,onlytextwidth]
    \begin{column}{0.43\textwidth}
        \vspace{-5pt}
        \begin{algorithm}[H]
        	\caption{Subsequenz 1}
        	\label{alg:subseq_part_1}
        	\DontPrintSemicolon
        	\Input{$A = a_1, \dots, a_n \in \nat$}
            \Output{Ist $A$ partitionierbar?}
        	{\color<2>{madderlake}$s \gets \frac{n \, (n+1)}{4}$\;}
        	\For{$k \gets 1$ \KwTo $n$}{
        	    \For{$\ell \gets k$ \KwTo $n$}{
        	        {\color<3>{madderlake}$s' \gets 0$\;}
        	        \For{$i \gets k$ \KwTo $\ell$}{
        	            {\color<5>{madderlake}$s' \gets s' + a_i$\;}
                    }
                    \If{{\color<3>{madderlake}$s' = s$}}{
                        {\color<4>{madderlake}\Return{\True}}
                    }
                }
            }
            {\color<4>{madderlake}\Return{\False}}
        \end{algorithm}
    \end{column}
    \begin{column}<2->{0.51\textwidth}
        \only<2->{
            Die \red<2>{Initialisierung} ben\"otigt 4 Instruktionen
            \begin{itemize}
                \item Eine Addition, eine Multiplikation, eine Division und eine Zuweisung
            \end{itemize}
        }
        
        \only<3->{
            Die \red<3>{Zuweisung} in Zeile $4$ und der \red<3>{Vergleich} in Zeile $6$ werden h\"ochstens $\frac{n \, (n+1)}{2}$-mal ausgef\"uhrt
        }
        
        \medskip
        
        \only<4->{
            Einmalig wird ein Wert \red<4>{zurückgegeben}
        }
        
        \medskip
        
        \only<5->{
            Die \red<5>{innere \textbf{for}-Schleife} wird $\frac{n \, (n+1) \, (n+2)}{6}$-mal durchlaufen (Details: n\"achste Folie!)
            \begin{itemize}
                \item Entsprechend wird auch Zeile $6$ so oft ausgef\"uhrt
            \end{itemize}
        }
        
        \only<6->{
            Insgesamt $\leq \nicefrac{1}{3} \, n \, (n+1) \, (n+5) + 5$ Instruktionen
        }
    \end{column}
\end{columns}
\end{frame}

\begin{frame}{Details zur Laufzeitschranke}
    \begin{block}{Beobachtungen}
        \begin{itemize}
            \item<+-> Die innere \textbf{for}-Schleife wird stets $(\ell - k + 1)$-mal durchlaufen
            \begin{itemize}
                \item $\underbrace{\underbrace{1}_{\ell=1}, \underbrace{2}_{\ell=2}, \dots, \underbrace{n}_{\ell=n}}_{k=1}\text{-mal}, \underbrace{1, \dots, (n-1)}_{k=2}\text{-mal}, \dots, \underbrace{1, \dots, (n-n+1)}_{k=n}\text{-mal}$
            \end{itemize}
            \item<+-> Als Summe ergibt sich $$\sum_{i=1}^n \frac{i \, (i+1)}{2} = \frac{1}{2} \, \sum_{i=1}^n i \, (i+1) = \frac{1}{2} \, \left({\color<4>{madderlake} \sum_{i=1}^n i^2} + {\color<3->{mandarin} \sum_{i=1}^n i} \right)$$
            \item<+-> Wir wissen bereits, dass \orange{$\sum_{i=1}^n i = \frac{1}{2} \, n \, (n+1)$}
            \item<+-> Wir k\"onnen per vollst\"andiger Induktion zeigen, dass \red{$\sum_{i=1}^n i^2 = \frac{1}{6} \, n \, (n + 1) \, (2 \, n + 1)$}
        \end{itemize}
    \end{block}
\end{frame}

\begin{frame}{Details zur Laufzeitschranke (2)}
    \begin{lemma}\label{lem:nsquaressum}
    F\"ur alle $n \in \natpos$ gilt $\sum_{i=1}^n i^2 = \frac{1}{6} \, n \, (n + 1) \, (2 \, n + 1)$.
    \end{lemma}
    
    \begin{proof}
    Per Induktion.
    F\"ur $n=1$ gilt Lemma \ref{lem:nsquaressum} offensichtlich.
    F\"ur $n \leadsto n+1$ erhalten wir:
    \begin{align*}
        \sum_{i=1}^{n+1} i^2 = \sum_{i=1}^{n} i^2 + (n+1)^2 &\overset{!}{=} \frac{1}{6} \, n \, (n + 1) \, (2 \, n + 1) + (n+1)^2 \\
        &= \frac{1}{6} \, (n + 1) \, (n \, (2 \, n + 1) + 6 \, (n+1)) \\
        &= \frac{1}{6} \, (n + 1) \, (2 \, n^2 + 7 \, n + 6) \\
        &= \frac{1}{6} \, (n + 1) \, (n + 2) \, (2 \, n + 3)
    \end{align*}
    \end{proof}
\end{frame}

\begin{frame}{Details zur Laufzeitschranke (3)}
    Insgesamt erhalten wir f\"ur Algorithmus \ref{alg:subseq_part_1} unter Verwendung der \orange{gaußschen Summenformel} und \red{Lemma \ref{lem:nsquaressum}}:
    \begin{align*}
        \frac{1}{2} \, \left({\color{madderlake}\sum_{i=1}^n i^2} + {\color{mandarin}\sum_{i=1}^n i} \right) &= \frac{1}{2} \, \left({\color{madderlake}\frac{1}{6} \, n \, (n + 1) \, (2 \, n + 1)} + {\color{mandarin}\frac{1}{2} \, n \, (n+1)} \right) \\
        &= \frac{1}{2} \cdot \frac{1}{6} \, \left( n \, (n+1) \, (2\,n+1) + 3 \, n \, (n+1) \right) \\
        &= \frac{1}{2} \cdot \frac{1}{6} \, (2\,n+1+3) \, n \, (n+1) \\
        &= \frac{1}{2} \cdot \frac{1}{6} \cdot 2\, (n+2) \, n \, (n+1) \\
        &= \frac{1}{6} \, n \, (n+1) \, (n+2)
    \end{align*}
\end{frame}

\begin{frame}{Bemerkung zum Algorithmus}
    \begin{remark}
        In Algorithmus \ref{alg:subseq_part_1} summieren wir viele Subsequenzen mehrfach.
        
        Sei etwa $A = 4, 2, 3, 1$ mit $n=|A|=4$ die Eingabe.
        Zu Beginn ist $k=1$ und $\ell=1..4$, es werden also nacheinander die Summen $4$, $4+2$, $4+2+3$ und $4+2+3+1$ bestimmt.
        
        Tats\"achlich kann aber jede der Summen durch \alert{eine einzelne Addition} zum Ergebnis der vorherigen Summe bestimmt werden.
        
        So kann die Laufzeit von $\leq \nicefrac{1}{3} \, n \, (n+1) \, (n+5) + 5$ auf $\leq \nicefrac{1}{2} \, (3 \, n^2 + 5 \, n + 10)$ reduziert werden.
    \end{remark}
\end{frame}

\againframe<3->{second_example}

\begin{frame}{Algorithmus \& Analyse}
\begin{columns}[T,onlytextwidth]
\begin{column}{0.43\textwidth}
\vspace{-5pt}
\begin{algorithm}[H]
	\caption{Subsequenz 2}
	\label{alg:subseq_part_2}
	\DontPrintSemicolon
	\Input{$A = a_1, \dots, a_n \in \nat$}
    \Output{Ist $A$ partitionierbar?}
	$s \gets \frac{n \, (n+1)}{4}$\;
	\For{$k \gets 1$ \KwTo $n$}{
	    $s' \gets 0$\;
	    \For{$\ell \gets k$ \KwTo $n$}{
	        $s' \gets s' + a_i$\;
	        
            \If{$s' = s$}{
                \Return{\True}
            }
        }
    }
    \Return{\False}
\end{algorithm}
\end{column}
\begin{column}{0.51\textwidth}
    Die Initialisierung ben\"otigt 4 Instruktionen
    
    \medskip
    
    Der Rumpf der \"au{\ss}eren \textbf{for}-Schleife, und somit auch die Zuweisung in Zeile 3, wird h\"ochstens $n$-mal ausgef\"uhrt
    
    \medskip
    
    Der Rumpf der inneren \textbf{for}-Schleife, bestehend aus einer Addition, einer Zuweisung und einem Vergleich in den Zeilen 5 und 6, wird h\"ochstens $\frac{n \, (n+1)}{2}$-mal ausgef\"uhrt
    
    \medskip
    
    Einmalig wird ein Wert zur\"uckgegeben
    
    \medskip
    
    Insgesamt $\leq \nicefrac{1}{2} \, (3 \, n^2 + 5 \, n + 10)$ Instruktionen
\end{column}
\end{columns}
\end{frame}

\begin{frame}{Details zur Laufzeitschranke}
    Die gegebene Laufzeitschranke von $\nicefrac{1}{2} \, (3 \, n^2 + 5 \, n + 10)$ erhalten wir wie folgt:
    \begin{align*}
        4 + n + 3 \cdot \frac{n \, (n+1)}{2} + 1 &= \nicefrac{1}{2} \cdot 3 \, n \, (n+1) + n + 5 \\
        &= \nicefrac{1}{2} \cdot (3 \, n^2 + 3\, n) + n + 5 \\
        &= \nicefrac{1}{2} \cdot (3 \, n^2 + 3\, n) + \nicefrac{1}{2} \cdot (2 \, n + 10) \\
        &= \nicefrac{1}{2} \cdot (3 \, n^2 + 5\, n + 10)
    \end{align*}

    \begin{remark}
    Insbesondere haben wir damit gezeigt, dass Algorithmus \ref{alg:subseq_part_2} \alert{asymptotisch effizienter} ist als Algorithmus \ref{alg:subseq_part_1}, da die Laufzeit von Algorithmus \ref{alg:subseq_part_2} in $\bigO(n^2)$, die von Algorithmus \ref{alg:subseq_part_1} hingegen in $\bigO(n^3)$ liegt.
    \end{remark}
\end{frame}

    \section{Komplexit\"atsanalyse}

\begin{frame}{Hintergrund}
\begin{block}{Knappheit von Ressourcen}
    \begin{itemize}
        \item Speicherplatz (Register, RAM, HDD/SSD, Tape, ...)
        \item \alert{Rechenleistung} (Taktfrequenz)
    \end{itemize}
\end{block}

\begin{block}{Eine Problemstellung -- viele Lösungsansätze}
    \begin{itemize}
        \item Welchen Ansatz sollen wir nutzen?
        \item Ziel: Klassifiziere Lösungsansätze nach ihrer \alert{Eignung}
        \begin{itemize}
            \item Eignung $\approx$ Ressourcenbedarf
        \end{itemize}
    \end{itemize}
\end{block}
\end{frame}

\begin{frame}{Berechnungsmodell}
Wir analysieren \alert{abstrakte Algorithmen} -- folglich st\"utzen wir unsere Analysen auf \alert{abstrakte Maschinen}.

\medskip

\begin{columns}[T,onlytextwidth]
\begin{column}{0.54\textwidth}
Beispielsweise ist die \alert{Random Access Machine} eine zur \alert{Turingmaschine} \"aquivalente und f\"ur unsere Zwecke leichter zu handhabende Abstraktion einen realen Computers.
\end{column}
\begin{column}{0.4\textwidth}
\begin{tikzpicture}

\end{tikzpicture}
\end{column}
\end{columns}
\end{frame}

\begin{frame}<1-2>[label=first_example]{Ein einfaches Beispiel}
\begin{example}
Gegeben sei eine Liste natürlicher Zahlen $a_1, \dots, a_n \in \nat$.

Maximiere $x = a_i \cdot a_j$ mit $1 \leq i < j \leq n$.

\begin{quote}
    In Worten: Finde das \alert{gr\"o{\ss}te} Produkt, das sich aus zwei \alert{verschiedenen} Elementen der Liste bilden lässt.
\end{quote}

\begin{block}<2->{M\"ogliche Ans\"atze}
    \begin{enumerate}
        \item Betrachte alle möglichen Paare und berechne deren Produkt
        \pause
        \item<4-> Suche die beiden gr\"o{\ss}ten Elemente $a_i, a_j$ und gib $a_i \cdot a_j$ aus
    \end{enumerate}
\end{block}
\end{example}
\end{frame}

\begin{frame}
\frametitle<1>{Algorithmus}
\frametitle<2>{Algorithmus \& Analyse}
\begin{columns}[T,onlytextwidth]
    \begin{column}{0.42\textwidth}
        \vspace{-5pt}
        \begin{algorithm}[H]
        	\caption{Maximales Produkt 1}
        	\label{alg:max_prod}
        	\DontPrintSemicolon
        	\Input{$a_1, \dots, a_n \in \nat$}
            \Output{$\max \{ a_i \cdot a_j : 1 \leq i < j \leq n \}$}
        	$x \gets 0$\;
        	\For{$i \gets 1$ \KwTo $n$}{
        	    \For{$j \gets i + 1$ \KwTo $n$}{
        	        \If{$a_i \cdot a_j > x$}{
        	            $x \gets a_i \cdot a_j$\;
        	        }
                }
            }
            \Return{$x$}
        \end{algorithm}
    \end{column}
    \begin{column}<2>{0.52\textwidth}
        Die \"au{\ss}ere \textbf{for}-Schleife wird $n$-mal durchlaufen, die innere \textbf{for}-Schleife $n-1, n-2, \dots, 0$-mal
        \begin{itemize}
            \item Anwendung der \alert{gaußschen Summenformel} liefert $\frac{n \, (n-1)}{2}$
        \end{itemize}
        
        Der Aufwand des \textbf{if}-Statements ist von $n$ \alert{unabh\"angig}
        \begin{itemize}
            \item Multiplikationen, Vergleiche und Zuweisungen ben\"otigen lediglich \alert{konstant} viel Zeit
        \end{itemize}
        
        Insgesamt $\leq 4 \cdot \frac{n \, (n-1)}{2} + 2$ Instruktionen
    \end{column}
\end{columns}
\end{frame}

\againframe<3->{first_example}

\begin{frame}
\frametitle<1>{Algorithmus}
\frametitle<2>{Algorithmus \& Analyse}
\begin{columns}[T,onlytextwidth]
    \begin{column}{0.44\textwidth}
        \vspace{-5pt}
        \begin{algorithm}[H]
        	\caption{Maximales Produkt 2}
        	\label{alg:max_prod_ii}
        	\DontPrintSemicolon
        	\Input{$a_1, \dots, a_n \in \nat$}
            \Output{$\max \{ a_i \cdot a_j : 1 \leq i < j \leq n \}$}
        	$k \gets 0, \quad \ell \gets 0$\;
        	\For{$i \gets 1$ \KwTo $n$}{
        	    \If{$a_i > a_k$}{
        	        $\ell \gets k, \quad k \gets i$\;
    	        }
    	        \ElseIf{$a_i > a_{\ell}$}{
    	            $\ell \gets i$\;
    	        }
            }
            \Return{$a_k \cdot a_{\ell}$}
        \end{algorithm}
    \end{column}
    \begin{column}<2>{0.5\textwidth}
        Die \textbf{for}-Schleife wird genau $n$-mal durchlaufen
        
        \medskip
        
        Innerhalb der \textbf{if}-Statements f\"uhren wir lediglich grundlegende Operationen wie Vergleiche und Zuweisungen durch
        
        \medskip
        
        Insgesamt $\leq 3 \, n$ Instruktionen
    \end{column}
\end{columns}
\end{frame}

\begin{frame}{\MakeUppercase{$\bigO$}-Notation}
\begin{columns}[T,onlytextwidth]
\begin{column}{0.47\textwidth}
    \begin{definition}[$\bigO$-Notation]
    Seien $\teal{f}, \red{g}: \real \to \real$ Funktionen.
    Wir schreiben $\teal{f} \in \bigO(\red{g})$ genau dann, wenn es eine Konstante $\orange{c} \in \realpos$ gibt, sodass ab einem $n_0$ f\"ur alle $n \geq n_0$ gilt, dass $\teal{f}(n) \leq \orange{c} \cdot \red{g}(n)$.
    \end{definition}
    
    \medskip
    
    {\footnotesize Das hei{\ss}t, ab einem gewissen Punkt $n_0$ gibt es keinen \alert{wesentlichen} Unterschied mehr zwischen den Funktionswerten von $\teal{f}$ und $\red{g}$.}
\end{column}
\begin{column}{0.47\textwidth}
    \begin{tikzpicture}
        \begin{axis}[tikzDefaults, xmin=-4, xmax=12, ymin=0, ymax=500]
            \foreach \k in {1, ..., 8}{
                \addplot+[domain=-4:12, draw=mandarin!40, line width=0.5pt, forget plot]{1/2 * \k * x^2};
            }
            \addplot+[domain=-4:12, draw=teal]{2*x^2 - 7*x + 18};
            \addlegendentry{$2 \, x^2 - 7 \, x + 18$}
            \addplot+[domain=-4:12, draw=madderlake]{x^2};
            \addlegendentry{$x^2$}
        \end{axis}
    \end{tikzpicture}
\end{column}
\end{columns}
\end{frame}

\begin{frame}{Beispiel}
\begin{example}
Sei \teal{$f(n) \coloneqq 3 \, n + 5$} und \red{$g(n) \coloneqq 2 \, n$}.

Um zu zeigen, dass $\teal{f} \in \bigO(\red{g})$ ist, w\"ahlen wir \orange{$c=2$} und erhalten
\begin{equation*}
    \teal{\only<1>{3 \, n} \only<2->{\cancel{3 \, n}} + 5} \leq \orange{2} \cdot \red{2 \, n} = \only<1>{3 \, n} \only<2->{\cancel{3 \, n}} + n.
\end{equation*}
\only<2->{F\"ur $n \geq 5$ ist also $\teal{f(n)} \leq \orange{2} \cdot \red{g(n)}$ und somit $n_0 = 5$.}
\end{example}
\end{frame}

\begin{frame}{Bemerkungen}
\begin{remark}<+->
In der Praxis schreibt man h\"aufig $f(n) \in \bigO(g(n))$ anstelle von $f \in \bigO(g)$.

Beispielsweise ist $$\teal{2 \, n^2 + n} \in \bigO(\red{n^2})$$ \"aquivalent zu $$\teal{f} \in \bigO(\red{g}) \text{ f\"ur } \teal{f(n) \coloneqq 2 \, n^2 + n} \text{ und } \red{g(n) \coloneqq n^2}.$$
\end{remark}

\begin{remark}<+->
F\"ur Algorithmus \ref{alg:max_prod} konnten wir zeigen, dass dieser h\"ochstens $4 \cdot \frac{n \, (n-1)}{2} + 2 \in \bigO(n^2)$ Instruktionen ausf\"uhrt.
Demgegen\"uber ben\"otigt Algorithmus \ref{alg:max_prod_ii} nur h\"ochstens $3 \, n \in \bigO(n)$ Instruktionen und ist somit \alert{asymptotisch} effizienter ist als Algorithmus \ref{alg:max_prod}.
\end{remark}
\end{frame}

\begin{frame}{Eine leichte Aufgabe}
\begin{task}<1->
Sei $f(n) \coloneqq 6 \, n + 7$.

Zeigen Sie, dass $f(n) \in \bigO(n)$ ist, d.h. finden Sie $c \in \realpos, n_0 \in \real$, sodass f\"ur alle $n \geq n_0$ gilt, dass $f(n) \leq c \cdot n$.
\end{task}

\begin{solution}<2->
F\"ur $c = 7$ ist $6 \, n + 7 \leq 7 \, n = 6 \, n + n$ f\"ur alle $n \geq n_0 \coloneqq 7$.
\end{solution}
\end{frame}

\begin{frame}{Mehr Aufgaben}
\begin{task}
Beweisen oder widerlegen Sie:
\begin{subtasks}
    \item $12 \, n - 4 \in \bigO(n)$
    \item $7 \, n + 1 \in \bigO(n^2)$
    \item $\log(n^2) \in \bigO(\log n)$
\end{subtasks}
\end{task}
\end{frame}

\begin{frame}{Alternative Definition}
\begin{definition}[$\bigO$-Notation]
Seien $\teal{f}, \red{g}: \real \to \real$ Funktionen.
Wir schreiben $\teal{f} \in \bigO(\red{g})$ genau dann, wenn $\limsup_{n \to \infty} \frac{\teal{f(n)}}{\red{g(n)}} < \infty$ und $\red{g(n)} > 0$ f\"ur alle $n \in \nat$.
\end{definition}

Das folgende Theorem liefert ein bei dieser Charakterisierung h\"aufig sehr n\"utzliches Werkzeug.

\begin{theorem}[Satz von L'Hospital]
...
\end{theorem}
\end{frame}

\begin{frame}{Ein Beispiel zum Satz von L'Hospital}
\begin{example}

\end{example}
\end{frame}

\begin{frame}{Aufgaben zum Satz von L'Hospital}
\begin{task}

\end{task}
\end{frame}

\begin{frame}{Aufgabe zur Analyse von Algorithmen}
\begin{task}

\end{task}
\end{frame}

    \section{Stacks \& Queues}

\begin{frame}{Abstrakte Datentypen}
\begin{block}{Grundlagen}
\begin{itemize}
    \item Ein \alert{abstrakter Datentyp (ADT)} beschreibt eine Menge von Operationen auf einer Menge von Werten
    \item Wir interessieren uns nur f\"ur die \alert{Signaturen} dieser Operationen
\end{itemize}
\end{block}

\begin{block}{Stacks \& Queues}
\begin{itemize}
    \item Eine \alert{Queue} (Warteschlange) ist ein ADT nach dem \alert{FIFO} (first-in, first-out) Prinzip
    \begin{itemize}
        \item Typische Operationen sind \texttt{enqueue} und \texttt{dequeue}
    \end{itemize}
    \item Ein \alert{Stack} (Stapel) ist ein ADT nach dem \alert{LIFO} (last-in, first-out) Prinzip
    \begin{itemize}
        \item Typische Operationen sind \texttt{push} und \texttt{pop}
    \end{itemize}
    \item Oft steht eine weitere Operation \texttt{head} zur Verf\"ugung, um das "n\"achste" Element zu betrachten, ohne den \alert{Zustand} der Datenstruktur zu modifizieren
\end{itemize}
\end{block}
\end{frame}

\begin{frame}[label=logexpr_intro]{Logische Ausdr\"ucke}
Im Folgenden f\"uhren wir \alert{Syntax} und \alert{Semantik} logischer Ausdr\"ucke ein
\begin{itemize}
    \item Die \alert{Syntax} beschreibt, \alert{wie} ein logischer Ausdruck aufgebaut sein muss
    \item Die \alert{Semantik} beschreibt, \alert{was} ein logischer Ausdruck aussagt
\end{itemize}

\begin{example}<2->
Wie wir gleich sehen werden, ist $\logic{(1 \land (1 \lor 0))}$ ein logischer Ausdruck.
\begin{itemize}
    \item Hierbei nehmen $\logic{1}$ und $\logic{0}$ die Rollen von "\emphteal{wahr}" und "\emphteal{falsch}" ein
    \item Um ihn "auszurechnen", m\"ussen wir u.a. festlegen, was das Ergebnis der Operatoren $\logic{\land}$ und $\logic{\lor}$ sein soll
    \item Intuitiv ist $\logic{(1 \land (1 \lor 0))} = \logic{(1 \land 1)} = \logic{1}$
\end{itemize}
\end{example}
\end{frame}

\begin{frame}{Syntax logischer Ausdr\"ucke}
\begin{definition}\label{sq:def:logexp_syn}
Wir definieren \emphteal{logische Ausdr\"ucke} als Worte \"uber dem Alphabet $$\Sigma = \{ \overbrace{\logic{0}, \, \logic{1}}^{\text{Ziffern}} \} \cup \{ \overbrace{\logic{\land}, \, \logic{\lor}, \, \logic{\oplus}}^{\text{Operatoren}}\} \cup \{ \overbrace{\logic{(}, \, \logic{)}}^{\text{Klammern}} \}.$$

Die Menge der logischen Ausdr\"ucke $\mathcal{L} \subseteq \Sigma^{\star}$ ist rekursiv wie folgt definiert:
\begin{itemize}
    \item Ziffern sind logische Ausdr\"ucke, d.h. $\{ \logic{0}, \logic{1} \} \subseteq \mathcal{L}$
    \item Falls $\teal{w_1}, \teal{w_2} \in \mathcal{L}$, so sind auch $\teal{(w_1 \land w_2)}, \teal{(w_1 \lor w_2)}, \teal{(w_1 \oplus w_2)} \in \mathcal{L}$
\end{itemize}
\end{definition}
\begin{example}<2->
Beispielsweise sind $\logic{1}$, $\logic{(0 \land 1)}$ und $\logic{(1 \land (1 \lor 0))}$ logische Ausdr\"ucke, w\"ahrend $\mathtt{\red{1 \land 1}}$ und $\mathtt{\red{(1 \land 1 \lor 0)}}$ \emph{keine} logischen Ausdr\"ucke sind.
\end{example}
\end{frame}

\begin{frame}{\"Uberpr\"ufung logischer Ausdr\"ucke}
\begin{task}
Sei $w \in \Sigma^{\star}$ mit $\Sigma = \{ \mathtt{0}, \mathtt{1} \} \cup \{ \mathtt{\land}, \mathtt{\lor}, \mathtt{\oplus} \} \cup \{ \mathtt{(}, \mathtt{)} \}$.
Entwerfen Sie einen Algorithmus, der bei Eingabe von $w$ in Zeit $\bigO(n)$ entscheidet, ob $w \in \mathcal{L}$ ist, wobei $n = |w|$ die L\"ange von $w$ in Zeichen sei.
\end{task}
\begin{solution}<2->
Wir arbeiten uns entsprechend der Klammerung von $w$ durch die Eingabe und nutzen den \alert{Call Stack}, um den Kontext vor Auswertung eines Teilausdrucks zu sichern
\begin{itemize}
    \item Dabei gehen wir gem\"a{\ss} der Syntax logischer Ausdr\"ucke vor, verlangen also, dass entweder $w \in \{ \mathtt{0}, \mathtt{1} \}$ oder $w = (w_1 \star w_2)$ mit $w_1, w_2 \in \mathcal{L}$ und $\star \in \{ \land, \lor, \oplus \}$
\end{itemize}
Beachte, dass wir $w$ dabei \emph{nicht} auswerten.
\end{solution}
\end{frame}

\begin{frame}{Algorithmus}
\begin{algorithm}[H]
	\caption{\"Uberpr\"ufung logischer Ausdr\"ucke}
	\label{sq:alg:check}
	\DontPrintSemicolon
    \SetKwFunction{check}{isExpression}
    \SetKwFunction{fsm}{fsm}
    \SetKwProg{Procedure}{procedure}{}{}
    \begin{multicols}{2}
    \Input{$w = w_1w_2\dots w_n \in \Sigma^{\star}$}
    \Output{$w \in \mathcal{L}$?}
    \BlankLine
    \Procedure{\check{$k$}}{
        \lIf{$w_k \in \{ \logic{0}, \logic{1} \}$}{
            \Return{$k$}
        }
        \lElseIf{$w_k = \logic{(}$}{
            \Return{\fsm{$k$}}
        }
        \lElse{\Return{$\bot$}}
    }
    \BlankLine
    \Return{\check{$1$} $= n$}\;
    \Procedure{\fsm{$k$}}{
        $s \gets 0$\;
        \For{$i \gets k + 1$ \KwTo $n$}{
            $\ell \gets$ \check{$i$}\;
            \If{$s \in \{ 0, 2\}$ \And $\ell > 0$}{
                $s \gets s + 1, \quad i \gets \ell$\;
            }
            \ElseIf{$s = 1$ \And $w_i \in \{ \logic{\land}, \logic{\lor}, \logic{\oplus} \}$}{
                $s \gets s + 1$\;
            }
            \ElseIf{$s = 3$ \And $w_i = \logic{)}$}{
                \Return{$i$}\;
            }
            \lElse{\Return{$\bot$}}
        }
    }
    \end{multicols}
\end{algorithm}
\end{frame}

% \begin{frame}{Analyse}
% \end{frame}

\begin{frame}{Alternativer Ansatz I}
\begin{block}{Grundlagen}
Wir k\"onnen die Sprache aller logischen Ausdr\"ucke $\mathcal{L}$ leicht durch Angabe einer \alert{kontextfreien Grammatik} spezifizieren
\begin{itemize}
    \item Dementsprechend k\"onnen wir einen \alert{Kellerautomaten} (PDA) angeben, der f\"ur Worte $w \in \Sigma^{\star}$ entscheidet, ob $w \in \mathcal{L}$
    \begin{itemize}
        \item Zu Details siehe Vorlesung \emph{Einf\"uhrung in die Theoretische Informatik}
    \end{itemize}
\end{itemize}
\end{block}

\begin{block}{Idee}<2->
Wir simulieren den Lauf eines PDA algorithmisch
\begin{itemize}
    \item Um einen PDA zu simulieren, ben\"otigen wir einen \alert{Stack}
    \item Die \alert{\"Uberf\"uhrungsfunktion} implementieren wir durch eine \alert{Lookup-Tabelle}
\end{itemize}

Nun k\"onnen wir jedes Zeichen der Eingabe in Zeit $\bigO(1)$ verarbeiten --- insgesamt $\bigO(n)$
\end{block}
\end{frame}

\begin{frame}[fragile]{Alternativer Ansatz II}
\begin{example}
{
\newcommand{\transition}[3]{\logic{#1}\purple{#2}{\color{hublue},}\,\purple{#3}}
\newcommand{\hash}{\text{\#}}
Folgender PDA $M = (Z, \teal{\Sigma}, \purple{\Gamma}, \delta, q_0, \purple{\text{\#}})$ mit $Z=\{q_0, q_1, q_2, q_3, q_4\}$ und $\purple{\Gamma = \{ \text{\#}, L, R \}}$ erkennt genau die logischen Ausdr\"ucke gem\"a{\ss} Definition \ref{sq:def:logexp_syn}:

\begin{figure}[!h]
\centering
\begin{tikzpicture}[node distance=2.2cm, every loop/.style={looseness=5, out=115, in=65}]
    \node[state, initial] (q0) {$q_0$};
    \node[state, right of=q0] (q1) {$q_1$};
    \node[state, right of=q1] (q2) {$q_2$};
    \node[state, right of=q2] (q3) {$q_3$};
    \node[state, right of=q3] (q4) {$q_4$};
    \draw   (q0) edge[loop above] node[align=center]{$\transition{0}{\hash}{\epsilon}$ \\ $\transition{1}{\hash}{\epsilon}$} (q0)
            (q0) edge[above] node{$\transition{(}{\hash}{L}$} (q1)
            (q1) edge[loop above] node{$\transition{(}{L}{LL}$} (q1)
            (q1) edge[above] node[align=center]{$\logic{0}$ \\ $\logic{1}$} (q2)
            (q2) edge[above] node[align=center]{$\transition{\land}{L}{R}$ \\ $\transition{\lor}{L}{R}$ \\ $\transition{\oplus}{L}{R}$} (q3)
            (q3) edge[bend left, above] node[align=center]{$\logic{0}$ \\ $\logic{1}$} (q4)
            (q3) edge[bend left, below] node{$\transition{(}{R}{RL}$} (q1)
            (q4) edge[bend left, below] node[align=center]{$\transition{\land}{L}{R}$ \\ $\transition{\lor}{L}{R}$ \\ $\transition{\oplus}{L}{R}$} (q3)
            (q4) edge[loop above] node{$\transition{)}{R}{\epsilon}$} (q4);
\end{tikzpicture}
\caption{Ein PDA $M$ mit $L(M) = \mathcal{L}$}
\label{sq:img:pda}
\end{figure}
}
\end{example}
\end{frame}

\againframe<3->{logexpr_intro}

\begin{frame}{Semantik logischer Ausdr\"ucke}
\begin{definition}
Wir definieren den \emph{Wert} $\llbracket \teal{w} \rrbracket$ logischer Ausdr\"ucke $\teal{w} \in \mathcal{L}$ rekursiv wie folgt:
\begin{itemize}
    \item $\eval{\logic{0}} = \logic{0}$ und $\eval{\logic{1}} = \logic{1}$
    \item F\"ur logische Ausdr\"ucke $\teal{w_1}, \teal{w_2}$ ist
    \begin{itemize}
        \item $\eval{\teal{(w_1 \land w_2)}} = \eval{\teal{w_1}} \cdot \eval{\teal{w_2}}$
        \item $\eval{\teal{(w_1 \lor w_2)}} = \begin{cases}\logic{1}, & \text{falls } \eval{\teal{w_1}} + \eval{\teal{w_2}} \geq 1 \\ \logic{0}, & \text{sonst.}\end{cases}$
        \item $\eval{\teal{(w_1 \oplus w_2)}} = \begin{cases}\logic{1}, & \text{falls } \eval{\teal{w_1}} \neq \eval{\teal{w_2}} \\ \logic{0}, & \text{sonst.}\end{cases}$
    \end{itemize}
\end{itemize}
\end{definition}
\begin{example}<2->\label{sq:ex:sem}
Sei $\logic{((0 \lor 1) \land (1 \land 0))}$ gegeben.
Wir erhalten $\eval{
\logic{(}
\underbrace{\logic{(0 \lor 1)}}_{\eval{\logic{(0 \lor 1)}} = \logic{1}}
\logic{\land}
\underbrace{\logic{(1 \land 0)}}_{\eval{\logic{(1 \land 0)}} = \logic{0}}
\logic{)}
} = \logic{0}$.
\end{example}
\end{frame}

\begin{frame}{Auswertung logischer Ausdr\"ucke}
\begin{task}
Entwerfen Sie einen Algorithmus, der einen logischen Ausdruck $\teal{w} \in \mathcal{L}$ auswertet, also $\eval{\teal{w}}$ bestimmt.
Die Laufzeit soll in $\bigO(n)$ liegen, wobei $n = |\teal{w}|$ die L\"ange des logischen Ausdrucks in Zeichen sei.

Nebst einer konstanten Anzahl zus\"atzlicher Variablen primitiven Datentyps d\"urfen Sie eine Queue oder einen Stack benutzen.
\end{task}
\begin{solution}<2->
\begin{quote}
    Idee: Wir werten \teal{w} wie in Beispiel \ref{sq:ex:sem} ``von innen nach au{\ss}en'' aus und merken uns Zwischenergebnisse in einem \alert{Stack}.
\end{quote}

Details folgen!
\end{solution}
\end{frame}
    \input{sections/sorting}
    \section{Suchverfahren}

\begin{frame}{Grundlagen}
\begin{block}{Problemstellung}
Gegeben eine \alert{sortierte} Liste $L = \alglist{a_1, \dots, a_n}$ \"uber einem Wertebereich $D$ und ein beliebiges Element $a \in D$.

Wir wollen die Frage beantworten, ob ein $i \in \{ 1, \dots, n \}$ mit $a_i = a$ existiert.
\end{block}
\begin{block}{Ansatz}<2->
Wir sprechen \"uber \alert{sortierte} Listen, k\"onnen also die \alert{Transitivit\"at} der $\leq_D$-Relation nutzen.

Das bedeutet, dass wir nach Vergleich eines Listenelements $a_i \neq a$ nur noch die Teilliste $L[1, i-1]$ bzw. $L[i+1, n]$ betrachten m\"ussen:
$$L = \alglist{\underbrace{a_1, \dots, a_{i-1}}_{\leq a_i}, a_i, \underbrace{a_{i+1}, \dots, a_n}_{\geq a_i}}$$
\end{block}
\end{frame}

\begin{frame}{Suchverfahren}
In der Vorlesung wurden folgende Suchverfahren vorgestellt:
\begin{itemize}
    \item Binary Search
    \item Fibonacci Search
    \item Interpolation Search
\end{itemize}
Im Folgenden führen wir Schreibtischtests o.g. Verfahren durch.
\end{frame}
    %\section{Amortisierte Analyse}

\begin{frame}{Amortisation in der Wirtschaft}
\resizebox{!}{.8\textheight}{%
\begin{tikzpicture}
\begin{axis}[
    tikzDefaults,
    smooth,
	area style,
    xmin=1,
    xmax=19,
    ticks=none,
    axis x line=middle,
    set layers={background, main}
]
    \addplot+[draw=mandarin, fill=mandarin, fill opacity=0.2] coordinates {
        (1, 289)
        (2, 330)
        (3, 361)
        (4, 393)
        (5, 425)
        (6, 462)
        (7, 62)
        (8, 80)
        (9, 107)
        (10, 141)
        (11, 177)
        (12, 226)
        (13, 279)
        (14, 345)
        (15, 422)
        (16, 512)
        (17, 607)
        (18, 718)
    }
    \closedcycle;
    \addplot+[draw=teal, fill=teal, fill opacity=0.2] coordinates {
        (1, 128)
        (2, 129)
        (3, 126)
        (4, 127)
        (5, 122)
        (6, 127)
        (7, 135)
        (8, 143)
        (9, 148)
        (10, 151)
        (11, 165)
        (12, 169)
        (13, 174)
        (14, 182)
        (15, 196)
        (16, 204)
        (17, 219)
        (18, 222)
    }
    \closedcycle;
    \addplot+[draw=madderlake, fill=madderlake, fill opacity=0.2] coordinates {
        (1, -89)
        (2, -88)
        (3, -95)
        (4, -95)
        (5, -90)
        (6, -90)
        (7, -105)
        (8, -105)
        (9, -104)
        (10, -106)
        (11, -108)
        (12, -109)
        (13, -102)
        (14, -100)
        (15, -102)
        (16, -106)
        (17, -106)
        (18, -105)
    }
    \closedcycle;
\end{axis}
\end{tikzpicture}
}
\end{frame}

\begin{frame}{Selbstorganisierende Listen}

\end{frame}
\end{document}
