\section{Heaps}

\begin{frame}{Priorit\"atswarteschlangen}
\begin{block}<+->{Grundidee}
Wir wollen schnell auf das aktuell h\"ochstpriorisierte Element einer \alert{ver\"anderlichen} Liste zugreifen k\"onnen.

Genauer gesagt soll die gesuchte Datenstruktur folgende Operationen unterst\"utzen:
\begin{itemize}
    \item \alert{Finden} und \alert{Entfernen} des aktuell h\"ochstpriorisierten Elements
    \item \alert{Einf\"ugen} neuer Elemente
\end{itemize}
\end{block}

\begin{block}<+->{Problem}
F\"ur sortierte/unsortierte Arrays/Linked-Lists ist stets mindestens eine der gew\"unschten Operationen in $\red{\bigO(n)}$.
\end{block}

\begin{block}<+->{L\"osung}
Wir organisieren unsere Daten in einem \alert{Baum}, den wir durch ein \alert{Array} repr\"asentieren.
\end{block}
\end{frame}

\begin{frame}{Heaps}
\begin{definition}\label{heaps:def:heap}\vspace*{-6pt}
Ein \alert{Heap} ist ein \emph{beinahe vollst\"andiger Bin\"arbaum} $T = (V, E, r)$ mit Beschriftungsfunktion $\pi: V \to \nat$, f\"ur den gilt:
\begin{quote}
    Ist $\purple<2->{v} \in V$ ein innerer Knoten mit Kindern $\teal<2->{v_{\ell}}, \red<2->{v_r}$, so ist $\pi(\purple<2->{v}) \leq \pi(\teal<2->{v_{\ell}})$ und $\pi(\purple<2->{v}) \leq \pi(\red<2->{v_r})$.
\end{quote}
\end{definition}

\begin{block}<2->{Skizze zu Definition \ref{heaps:def:heap}}
\begin{tikzpicture}[every edge/.append style={draw, -}]
    \node[amaranth, fill=amaranth!7] (A) at (0,0) {4};
    \node[tealblue, fill=tealblue!7] (B) at (-1,-1) {7};
    \node[madderlake, fill=madderlake!7] (C) at (1,-1) {5};
    
    \node[draw=none] (D) at (-0.6,0.6) {};
    \node[draw=none] (E) at (-1.6,-1.6) {};
    \node[draw=none] (F) at (-0.4,-1.6) {};
    \node[draw=none] (G) at (0.4,-1.6) {};
    \node[draw=none] (H) at (1.6,-1.6) {};
    
    \path (A) edge (B);
    \path (A) edge (C);
    \path (A) edge (D);
    \path (B) edge (E);
    \path (B) edge (F);
    \path (C) edge (G);
    \path (C) edge (H);
\end{tikzpicture}
\end{block}
\end{frame}

\begin{frame}{Operationen auf Heaps}
Im Folgenden betrachten wir Operationen zum Erstellen und Verwalten von Heaps:
\begin{itemize}
    \item \lightblue{\textsc{BottomUpSiftDown}} zum Aufbau einer Heap-Struktur
    \item \lightblue{\textsc{Add}} zum Einf\"ugen eines neuen Elements
    \item \lightblue{\textsc{ExtractMin}} zum Entfernen des minimalen Elements
\end{itemize}

\begin{remark}<2->
S\"amtliche o.g. Operationen lassen sich durch Anpassung einiger Vergleiche leicht auf sog. Max-Heaps \"ubertragen.
\end{remark}
\end{frame}

% \begin{frame}{Prequel: Die Heapify Operation}
% \begin{columns}[T]
% \begin{column}{0.4\textwidth}
% \begin{algorithm}[H]
% 	\caption{Heapify}
% 	\label{heap:alg:extractmin}
% 	\DontPrintSemicolon
% 	\Input{Dynamisches Array $A$}
%     \Output{Minimales Element aus $A$}
% 	$min \gets A_1$\;
% 	Vertausche $A_1$ und $A_n$\;
% 	L\"osche $A_n$\;
% 	\textsc{Heapify($A$)}\;
%     \Return{$min$}\;
% \end{algorithm}
% \end{column}
% \begin{column}{0.55\textwidth}

% \end{column}
% \end{columns}
% \end{frame}

% \begin{frame}{Entfernen des minimalen Elements}
% \begin{columns}[T]
% \begin{column}{0.4\textwidth}
% \begin{algorithm}[H]
% 	\caption{ExtractMin}
% 	\label{heap:alg:extractmin}
% 	\DontPrintSemicolon
% 	\Input{Dynamisches Array $A$}
%     \Output{Minimales Element aus $A$}
% 	$min \gets A_1$\;
% 	Vertausche $A_1$ und $A_n$\;
% 	L\"osche $A_n$\;
% 	\textsc{Heapify($A$)}\;
%     \Return{$min$}\;
% \end{algorithm}
% \end{column}
% \begin{column}{0.55\textwidth}
% Wir vertauschen also die Wurzel mit dem letzten Blatt, stellen anschlie{\ss}end die Heap-Eigenschaft wieder her und geben den Wert der vormaligen Wurzel zur\"uck.

% \medskip

% Die Laufzeit liegt somit offensichtlich \alert{amortisiert} in $\bigO(\log n)$.
% \begin{itemize}
%     \item Amortisiert deshalb, weil das L\"oschen des letzten Elements aus einem \emph{dynamischen} Array im Worst-Case $\bigO(n)$ kostet
% \end{itemize}
% \end{column}
% \end{columns}
% \end{frame}

% \begin{frame}{Einf\"ugen eines neuen Elements}
% \begin{columns}[T]
% \begin{column}{0.4\textwidth}
% \begin{algorithm}[H]
% 	\caption{Add}
% 	\label{heap:alg:add}
% 	\DontPrintSemicolon
% 	\Input{Dynamisches Array $A$,\\Neuer Wert $a$}
%     \Output{Dynamisches Array $A'$}
%     $A_{n+1} \gets a$\;
%     $c \gets n + 1, \quad p \gets \floor{\nicefrac{c}{2}}$\;
%     \While{$p > 0$ \And $A_c < A_p$}{
%         Vertausche $A_c$ und $A_p$\;
%         $c \gets p, \quad p \gets \floor{\nicefrac{c}{2}}$\;
%     }
% \end{algorithm}
% \end{column}
% \begin{column}{0.55\textwidth}
% Wir machen also praktisch "BubbleSort" auf dem \emph{eindeutigen} Pfad von dem neu eingef\"ugten Blatt mit Wert $a$ und der Wurzel.

% \medskip

% Da heapgeordnete Arrays beinahe vollst\"andig sind, ist die L\"ange dieses Pfades in $\bigO(\log n)$ und somit auch die Laufzeit des Algorithmus'.
% \end{column}
% \end{columns}
% \end{frame}

% \begin{frame}{Aufbau einer Heap-Struktur}
    
% \end{frame}