\section{Sortierverfahren}

\begin{frame}{Grundlagen}
\begin{block}{Listen}
\begin{itemize}
    \item Abstrakter Datentyp zur Repr\"asentation geordneter Sequenzen von Datenwerten
    \item Verschiedene Implementierungen m\"oglich, bspw. als \alert{Array} oder durch \alert{Verkettung}
\end{itemize}
\end{block}

\begin{block}{Vereinbarungen}
Unabh\"angig von der konkreten Implementierung nehmen wir stets an, dass
\begin{itemize}
    \item Listen eine endliche L\"ange $n \in \nat$ haben
    \item Elemente einer Liste durch die Positionen $1, \dots, n$ adressiert werden
    %\item Operationen \texttt{insert}, \texttt{delete} und \texttt{elementAt} zur Verf\"ugung stehen
\end{itemize}
Den Zugriff auf das $i$-te Element einer Liste $L$ notieren wir mit $L_i$.
Die Teilliste vom $i$-ten bis zum $j$-ten Element (mit $i \leq j$) bezeichnen wir mit $L[i, j]$.
\end{block}
\end{frame}

\begin{frame}<1-3>[label=general_sort]
\frametitle<1-3>{Allgemeine Sortierverfahren}
\frametitle<4>{Zur Erinnerung: Allgemeine Sortierverfahren}
\begin{definition}[Allgemeines Sortierverfahren]
Sei $L = \alglist{a_1, \dots, a_n}$ eine Liste \"uber dem durch $\leq_D$ linear geordneten Definitionsbereich $D$.

Ein \alert{\red<2>{allgemeines} Sortierverfahren} berechnet eine Liste $L' = \alglist{a_{\pi(1)}, \dots, a_{\pi(n)}}$ f\"ur eine Permutation $\pi$ von $\{ 1, \dots, n\}$, sodass
\begin{enumerate}
    \item $a_{\pi(i)} \leq_D a_{\pi(i+1)}$ f\"ur alle $1 \leq i < n$ und
    \item<2-> \emphred<2>{nebst $\leq_D$ keine weiteren evtl. auf $D$ definierten Operationen genutzt werden}.
\end{enumerate}
\end{definition}

\begin{remark}<3->
Obgleich wir \emph{oft} Listen \"uber den nat\"urlichen Zahlen $\nat$ betrachten, d\"urfen wir \emph{keinen} Gebrauch von Operationen wie etwa Multiplikation oder Test auf \alert{Kongruenz} machen.
\end{remark}
\pause
\end{frame}

\begin{frame}{\"Uberblick \"uber allgemeine Sortierverfahren}
In der Vorlesung wurden unter anderem folgende \emph{allgemeine} Sortierverfahren vorgestellt:
\begin{itemize}
    \item Bubble Sort
    \item Insertion Sort
    \item Merge Sort
    \item Quick Sort
\end{itemize}
Im Folgenden f\"uhren wir \emph{Schreibtischtests} o.g. Verfahren durch.
\end{frame}

\begin{frame}{Bubble Sort}
\begin{columns}[T,onlytextwidth]
\begin{column}{0.4\textwidth}
\begin{algorithm}[H]
	\caption{Bubble Sort}
	\label{sort:alg:bubblesort}
	\DontPrintSemicolon
	\Input{Liste $L$}
    \Output{Sortierte Liste $L'$}
	\For{$i \gets 1$ \KwTo $n$}{
	    \For{$j \gets 1$ \KwTo $n - i$}{
            \If{$L_j \geq L_{j+1}$}{
                $t \gets L_j$\;
                $L_j \gets L_{j+1}$\;
                $L_{j+1} \gets t$\;
            }
        }
    }
    \Return{$L$}\;
\end{algorithm}
\end{column}
\begin{column}{0.55\textwidth}
\begin{task}<2->
Ordnen Sie folgende Liste mittels des nebenstehenden Algorithmus: $$L = \alglist{17, 5, 4, 63, 46, 58, 4}$$
\end{task}
% \begin{solution}<3->

% \end{solution}
\end{column}
\end{columns}
\end{frame}

\begin{frame}{Insertion Sort}
\begin{columns}[T,onlytextwidth]
\begin{column}{0.4\textwidth}
\begin{algorithm}[H]
	\caption{Insertion Sort}
	\label{sort:alg:insertionsort}
	\DontPrintSemicolon
	\Input{Liste $L$}
    \Output{Sortierte Liste $L'$}
	\For{$i \gets 2$ \KwTo $n$}{
	    $key \gets L_i$\;
	    $j \gets i - 1$\;
	    \While{$L_j > key$ \And $j > 0$}{
            $L_{j+1} \gets L_j$\;
            $j \gets j - 1$\;
        }
    }
    \Return{$L$}\;
\end{algorithm}
\end{column}
\begin{column}{0.55\textwidth}
\begin{task}<2->
Ordnen Sie folgende Liste mittels des nebenstehenden Algorithmus: $$L = \alglist{17, 5, 4, 63, 46, 58, 4}$$
\end{task}
% \begin{solution}<3->

% \end{solution}
\end{column}
\end{columns}
\end{frame}

\begin{frame}{Mergen sortierter Arrays}
Seien $A = \alglist{a_1, \dots, a_m}$ und $B = \alglist{a_{m+1}, \dots, a_{m+n}}$ sortierte Listen der L\"ange $m$ bzw. $n$.
Wir wollen nun eine ebenfalls sortierte Liste $L = \alglist{ a_{\pi(1)}, \dots, a_{\pi(m + n)} }$ bestimmen.
\begin{itemize}
    \item Das geht nat\"urlich, indem wir die Listen $A$ und $B$ zu $L' = \alglist{a_1, \dots, a_m, a_{m+1}, \dots, a_{m+n}}$ konkatenieren und $L'$ dann in $\bigO((m + n) \cdot \log(m + n))$ sortieren
    \item Damit nutzen wir aber nicht, dass $A$ und $B$ bereits sortiert sind
\end{itemize}
Stattdessen durchlaufen wir beide Listen "abwechselnd" von vorn nach hinten und f\"ugen den jeweils kleineren Wert ans Ende eines neuen Arrays ein
\begin{itemize}
    \item Nachteil: Wir ben\"otigen $\bigO(m+n)$ zus\"atzlichen Speicherplatz
\end{itemize}
\end{frame}

\begin{frame}{Mergen sortierter Arrays --- Beispiel}
\begin{example}
Sei $A = \alglist{ 2, 3, 5, 7, 11, 12 }$ und $B = \alglist{ 4, 11, 16, 17 }$.

\medskip

\begin{tabular}[t]{lr}
$A$: &
\begin{tabular}[t]{r|L{0.4cm}|}
    \cline{2-2} 
    \only<1>{$\rightarrow$} & 2 \\ \cline{2-2}
    \only<2>{$\rightarrow$} & 3 \\ \cline{2-2}
    \only<3-4>{$\rightarrow$} & 5 \\ \cline{2-2}
    \only<5>{$\rightarrow$} & 7 \\ \cline{2-2}
    \only<6>{$\rightarrow$} & 11 \\ \cline{2-2}
    \only<7-11>{$\rightarrow$} & 12 \\ \cline{2-2}
\end{tabular}
\end{tabular}
\hfill
\begin{tabular}[t]{lr}
$B$: &
\begin{tabular}[t]{r|L{0.4cm}|}
    \cline{2-2} 
    \only<1-3>{$\rightarrow$} & 4 \\ \cline{2-2}
    \only<4-7>{$\rightarrow$} & 11 \\ \cline{2-2}
    \only<8-9>{$\rightarrow$} & 16 \\ \cline{2-2}
    \only<10-11>{$\rightarrow$} & 17 \\ \cline{2-2}
\end{tabular}
\end{tabular}
\hfill
\begin{tabular}[t]{lr}
$L$: &
\begin{tabular}[t]{r|L{0.4cm}|}
    \cline{2-2} 
    \only<1>{$\rightarrow$} & \only<2->{2} \\ \cline{2-2}
    \only<2>{$\rightarrow$} & \only<3->{3} \\ \cline{2-2}
    \only<3>{$\rightarrow$} & \only<4->{4} \\ \cline{2-2}
    \only<4>{$\rightarrow$} & \only<5->{5} \\ \cline{2-2}
    \only<5>{$\rightarrow$} & \only<6->{7} \\ \cline{2-2}
    \only<6>{$\rightarrow$} & \only<7->{11} \\ \cline{2-2}
    \only<7>{$\rightarrow$} & \only<8->{11} \\ \cline{2-2}
    \only<8>{$\rightarrow$} & \only<9->{12} \\ \cline{2-2}
    \only<9>{$\rightarrow$} & \only<10->{16} \\ \cline{2-2}
    \only<10-11>{$\rightarrow$} & \only<11->{17} \\ \cline{2-2}
\end{tabular}
\end{tabular}
\end{example}
\end{frame}

\begin{frame}{Mergen sortierter Arrays --- Algorithmus}
\begin{algorithm}[H]
	\caption{Merge}
	\label{sort:alg:merge}
	\DontPrintSemicolon
	\Input{Sortierte Listen $A = \alglist{a_1, \dots, a_m}$ und $B = \alglist{a_{m+1}, \dots, a_{m+n}}$}
    \Output{Sortierte Liste $L = \alglist{a_{\pi(1)}, \dots, a_{\pi(m+n)}}$}
    \BlankLine
    \begin{multicols}{2}
    $i \gets 1, \quad j \gets m + 1, \quad k \gets 1$\;
    \While{$i \leq m$ \And $j \leq m + n$}{
        \If{$A_i \leq B_j$}{
            $L_k \gets A_i, \quad i \gets i + 1$
        }
        \Else{
            $L_k \gets B_j, \quad j \gets j + 1$
        }
        $k \gets k + 1$\;
    }
    \If{$i > m$}{
        $L[k, m+n] \gets B[j, m+n]$
    }
    \Else{
        $L[k, m+n] \gets A[i, m]$
    }
    \Return{$L$}\;
    \end{multicols}
\end{algorithm}
\end{frame}

\begin{frame}{Merge Sort}
\begin{columns}[T,onlytextwidth]
\begin{column}{0.42\textwidth}
\begin{algorithm}[H]
	\caption{Merge Sort}
	\label{sort:alg:mergesort}
	\DontPrintSemicolon
	\Input{Liste $L$}
    \Output{Sortierte Liste $L'$}
    \SetKwFunction{MergeSort}{MergeSort}
    \SetKwFunction{Merge}{Merge}
    \SetKwProg{Procedure}{procedure}{}{}
    \Procedure{\MergeSort{$L$}}{
        $n \gets |L|, \quad m \gets \floor{\nicefrac{n}{2}}$\;
        \BlankLine
        \lIf{$n \leq 1$}{
            \Return{$L$}
        }
        \BlankLine
        $L_{\ell} \gets$ \MergeSort{$L[1, m]$}\;
        $L_r \gets$ \MergeSort{$L[m + 1, n]$}\;
        \BlankLine
        \Return{\Merge{$L_{\ell}$, $L_r$}}\;
    }
    \Return{\MergeSort{$L$}}\;
\end{algorithm}
\end{column}
\begin{column}{0.53\textwidth}
\begin{task}<2->
Ordnen Sie folgende Liste mittels des nebenstehenden Algorithmus: $$L = \alglist{17, 5, 4, 63, 46, 58, 4}$$
\end{task}
% \begin{solution}<3->

% \end{solution}
\end{column}
\end{columns}
\end{frame}

\begin{frame}{Partitionierung von Listen}
\begin{definition}[Partitionierung]
Wir nennen eine Liste $L = \alglist{a_1, \dots, a_n}$ \alert{partitioniert bzgl. $a_k$} f\"ur ein $k \in \{ 1, \dots, n \}$, falls $a_i \leq a_k$ f\"ur alle $i < k$ und $a_j \geq a_k$ f\"ur alle $j > k$:
$$L = \alglist{\underbrace{a_1, \dots, a_{k-1}}_{\leq a_k}, a_k, \underbrace{a_{k+1}, \dots, a_n}_{\geq a_k}}$$
\end{definition}

\begin{example}<2->
Die Liste $L = \alglist{4, 7, 3, \teal{8}, 12, 9, \red{8}, 11}$ ist partitioniert bzgl. $\teal{8}$, \emph{nicht} aber bzgl. $\red{8}$.
\end{example}
\end{frame}

\begin{frame}{Partitionierung von Listen --- Problemstellung}
\begin{block}{Problemstellung}
Gegeben eine Liste $L = \alglist{a_1, \dots, a_n}$ und ein $k \in \{ 1, \dots, n \}$.
Bestimme in Zeit $\bigO(n)$ eine Liste $L' = \alglist{a_{\pi(1)}, \dots, a_{\pi(n)}}$, die bzgl. $a_{\pi^{-1}(k)}$ partitioniert ist.
\end{block}

\begin{remark}<2->
Da wir sp\"ater auf den Teillisten $L'[1, \pi^{-1}(k)-1]$ und $L'[\pi^{-1}(k)+1, n]$ links und rechts von Position $\pi^{-1}(k)$ rekursiv absteigen werden, ist idealerweise $\pi^{-1}(k) \approx \nicefrac{n}{2}$.
\end{remark}

\begin{block}{Frage}<3->
Wie k\"onnen wir ein solches $k$ in Zeit h\"ochstens $\bigO(n)$ finden?
\end{block}

\begin{block}{Antwort}<4->
Das ist gleichbedeutend damit, den \alert{Median} von $L$ in Zeit $\bigO(n)$ zu bestimmen --- das ist m\"oglich, aber nicht trivial (siehe bspw. \href{https://www.cs.hhu.de/fileadmin/redaktion/Fakultaeten/Mathematisch-Naturwissenschaftliche_Fakultaet/Informatik/Computational_Cell_Biology/AlDat/Alg2016_03.pdf}{\alert{hier}})

Eine zuver\"assige \alert{Heuristik} ist es, einen ``zuf\"alligen'' Wert zu w\"ahlen --- bspw. immer $a_n$
\end{block}
\end{frame}

\begin{frame}{Partitionierung von Listen --- Algorithmus}
\begin{algorithm}[H]
	\caption{Partitionierung}
	\label{sort:alg:partition}
	\DontPrintSemicolon
	\SetKwFunction{Swap}{swap}
	\SetKwRepeat{Do}{do}{while}
	\Input{Liste $L = \alglist{a_1, \dots, a_n}$}
    \Output{Bzgl. $a_n$ partitionierte Liste $L' = \alglist{a_{\pi(1)}, \dots, a_{\pi(n)}}$}
    $i \gets 1, \quad j \gets n - 1$\;
    \Do{$i < j$}{
        \lWhile{$L_i \leq L_n$ \And $i < n$}{$i \gets i + 1$}
        \lWhile{$L_j \geq L_n$ \And $j > 1$}{$j \gets j - 1$}
        \lIf{$i < j$}{\Swap{$L_i$, $L_j$}}
    }
    \BlankLine
    \Swap{$L_i$, $L_n$}\;
    \BlankLine
    \Return{$L[1, i - 1], L[i + 1, n]$}\;
\end{algorithm}
\end{frame}

\begin{frame}{Quick Sort}
\begin{columns}[T,onlytextwidth]
\begin{column}{0.42\textwidth}
\begin{algorithm}[H]
	\caption{Quick Sort}
	\label{sort:alg:quicksort}
	\DontPrintSemicolon
	\Input{Liste $L$}
    \Output{Sortierte Liste $L'$}
    \SetKwFunction{QuickSort}{QuickSort}
    \SetKwFunction{Merge}{Merge}
    \SetKwProg{Procedure}{procedure}{}{}
    \Procedure{\QuickSort{$L$}}{
        \lIf{$|L| \leq 1$}{\Return{$L$}}
        $L^L, L^R \gets \text{partitioniere } L \text{ bzgl. } L_n$\;
        $L' \gets $ \QuickSort{$L^L$}\;
        $R' \gets $ \QuickSort{$L^R$}\;
    }
    \Return{$\alglist{L', L_n, R'}$}\;
\end{algorithm}
\end{column}
\begin{column}{0.53\textwidth}
\begin{task}<2->
Ordnen Sie folgende Liste mittels des nebenstehenden Algorithmus: $$L = \alglist{17, 5, 4, 63, 46, 58, 4}$$
\end{task}
% \begin{solution}<3->

% \end{solution}
\end{column}
\end{columns}
\end{frame}

% \begin{frame}{Problemreduktion}
% Problemreduktionen sind ein in der theoretischen Informatik h\"aufig genutztes Werkzeug, um per \alert{Widerspruch} zu zeigen, dass ein Problem \emph{nicht} in einer bestimmten Komplexit\"atsklasse liegt.

% \begin{block}{Grundidee}<2->
% Sei $a$ eine Eingabe f\"ur ein Problem $\mathcal{A}$ und analog $b$ eine Eingabe f\"ur ein Problem $\mathcal{B}$.

% Wenn wir 
% \begin{enumerate}
%     \item einen Algorithmus $\mathbb{B}$ haben, der eine Eingabe $b$ f\"ur $\mathcal{B}$ verarbeiten kann (kurz: $\mathbb{B}(b)$ \emph{und} 
%     \item einen Algorithmus $\mathbb{T}$, der eine Eingabe $a$ f\"ur $\mathcal{A}$ in eine Eingabe f\"ur $\mathcal{B}$ \alert{transformieren} kann,
% \end{enumerate}
% dann k\"onnen wir $b \coloneqq \mathbb{F}(a)$ als Eingabe f\"ur $\mathbb{B}$ nutzen und so 

% \end{block}

% \begin{example}<3->
% Siehe Tafel!
% \end{example}
% \end{frame}

\begin{frame}{Bemerkung zur Laufzeit}
In der Vorlesung wurde folgendes Theorem bewiesen:

\begin{theorem}\label{sort:thm:bound}
Sei $L = \alglist{a_1, \dots, a_n}$ eine Liste mit $a_i \neq a_j$ f\"ur $1 \leq i < j \leq n$.
Um $L$ zu sortieren, ben\"otigt jedes \emph{allgemeine Sortierverfahren} im \emph{Worst Case} $\Omega(n \, \log n)$ Vergleiche.
\end{theorem}

\begin{block}{Bedeutung}
Das bedeutet insbesondere, dass wir f\"ur jeden der vorgestellten Algorithmen Instanzen erzeugen k\"onnen, f\"ur deren Sortierung mindestens $n \, \log n$ Vergleiche ben\"otigt werden.

Das gilt aber nur, wenn wir von \emph{beliebigen} Listen als Eingabe ausgehen.
\begin{itemize}
    \item Andernfalls k\"onnen wir gelegentlich durch zus\"atzliche \emph{Annahmen \"uber die Beschaffenheit} der Liste die untere Schranke f\"ur \emph{beliebige} Listen unterbieten.
\end{itemize}
\end{block}
\end{frame}

\begin{frame}{Sortieren spezieller Arrays}
Gelegentlich besitzen wir zus\"atzliches Wissen \"uber die Beschaffenheit einer eingegebenen Liste $L$.
Beispiele daf\"ur sind etwa:
\begin{itemize}
    \item "$L$ ist so vorsortiert, dass alle Elemente der ersten Hälfte von $L$ kleiner sind als alle Elemente der zweiten Hälfte."
    \item "Die Länge von $L$ ist durch 10 teilbar, und jeweils 10 aufeinanderfolgende Elemente (vom Anfang beginnend) sind aufsteigend sortiert."
\end{itemize}
Hierbei stellen wir uns die Frage, inwiefern sich diese Eigenschaften ggf. zur Reduktion der Laufzeit f\"ur's Sortieren nutzen lassen --- etwa auf $\bigO(n)$.

\begin{alertblock}{Achtung}<2->
Die einzelnen Datenwerte d\"urfen wir aber nach wie vor \emphred{nur miteinander vergleichen}!
\end{alertblock}
\end{frame}

% \begin{frame}{Entscheidungsprobleme}
% \begin{definition}[Entscheidungsproblem]
% Sei $U$ eine Menge und sei $A \subseteq U$.
% Das Problem, für ein $x \in U$ zu \emph{entscheiden}, ob auch $x \in A$ ist, wird das \alert{Entscheidungsproblem} für $A$ genannt.

% Ein Algorithmus $\mathbb{A}: U \to \{ 0, 1 \}$ \alert{entscheidet} $A$, falls $\mathbb{A}(x) = 1 \iff x \in A$.

% Das Entscheidungsproblem f\"ur $A$ liegt in $\bigO(g)$ für eine Funktion $g$, falls ein Algorithmus existiert, der $A$ in Zeit $\bigO(g)$ entscheidet.
% \end{definition}

% \begin{example}<2->
% Sei $U = \nat$ die Menge \emph{aller} nat\"urlichen Zahlen und sei $A = \{ 2 \, x : x \in U \}$ die Menge der \emph{geraden} nat\"urlichen Zahlen.
% Die Frage, ob eine beliebige nat\"urliche Zahl $x \in U$ auch gerade ist (also ob $x \in A$), kann algorithmisch leicht in Zeit $\bigO(1)$ entschieden werden, indem wir $\ceil{\nicefrac{x}{2}} \cdot 2 \overset{?}{=} x$ auswerten und ggf. $1$ bzw. $0$ ausgeben.
% \end{example}

% \end{frame}

% \begin{frame}{Problemreduktion}
% Seien $A \subseteq U_1$ und $B \subseteq U_2$ und sei $A$ \emph{nicht} in $\bigO(g)$ entscheidbar.
% Wir wollen zeigen, dass $B$ \emph{ebenfalls nicht} in $\bigO(g)$ entschieden werden kann.

% Zwecks Erzeugung eines \alert{Widerspruchs} nehmen wir an, wir h\"atten einen Algorithmus $\mathbb{B}$, der $B$ in $\bigO(g)$ entscheidet.

% Wenn wir \emph{zeigen} können, dass wir jedes $a \in U_1$ mittels eines Algorithmus $\mathbb{T}$ in Zeit $\bigO(g)$ in ein $\mathbb{T}(a) \in U_2$ \alert{transformieren} können, sodass $a \in A \iff \mathbb{T}(a) \in B$, dann entscheidet folgender Algorithmus $A$ in Zeit $\bigO(g)$:
% \begin{align*}
%     &b \gets \mathbb{T}(a) \\
%     &\mathbf{return} \  \mathbb{B}(b)
% \end{align*}

% Das steht aber im Widerspruch dazu, dass $A$ \emph{nicht} in Zeit $\bigO(g)$ entschieden werden \emph{kann}.
% \end{frame}

\begin{frame}{Aufgaben zum Sortieren spezieller Arrays I}
\begin{task}
Sei $L$ eine Liste der L\"ange $n$ mit $L_i \leq L_{i + 2}$ f\"ur alle $1 \leq i \leq n - 2$.
Gibt es ein \emph{allgemeines Sortierverfahren}, das $L$ in Zeit $\bigO(n)$ sortiert?
\end{task}

\begin{solution}<2->
Betrachte die Teillisten $L' = \alglist{L_1, L_3, L_5, \dots}$ und $L'' = \alglist{L_2, L_4, L_6, \dots}$.
Diese k\"onnen wir in Zeit $\bigO(n)$ aus $L$ bestimmen und per Voraussetzung sind sie bereits sortiert.
Indem wir nun $L'$ und $L''$ in $\bigO(n)$ \emph{mergen}, haben wir $L$ in $\bigO(n)$ sortiert.
\end{solution}
\end{frame}

\begin{frame}{Aufgaben zum Sortieren spezieller Arrays II}
\begin{task}
Sei $L$ eine Liste der L\"ange $n$ mit $L_{\floor{\nicefrac{n}{2}} + 1} = L_{\floor{\nicefrac{n}{2}} + 2} = \dots = L_n$.
Gibt es ein \emph{allgemeines Sortierverfahren}, das $L$ in Zeit $\bigO(n)$ sortiert?
\end{task}

\begin{solution}<2->
Angenommen, es g\"abe einen Algorithmus $\mathbb{A}$, der \emph{solche} Listen in $\bigO(n)$ sortieren kann.
Dann k\"onnen wir in einer \alert{beliebigen} Liste $L$ der L\"ange $n$ einfach einen beliebigen Wert $a$ ans Ende anh\"angen -- und zwar $n$-mal -- und so in Zeit $\bigO(n)$ eine Liste $L'$ der L\"ange $2 \, n$ erhalten, die die Voraussetzungen von Algorithmus $\mathbb{A}$ erf\"ullt.

Wir sortieren nun $L'$ mit $\mathbb{A}$ in Zeit $\bigO(n)$ und l\"oschen anschlie{\ss}end die $n$ hinzugef\"ugten Vorkommen von $a$ wieder aus der sortierten Liste.

Somit haben wir $L$ in Zeit $\bigO(n)$ sortiert --- Widerspruch zu Theorem \ref{sort:thm:bound}!
\end{solution}
\end{frame}

\againframe<4>{general_sort}

\begin{frame}{Spezielle Sortierverfahren}
H\"aufig haben wir \alert{spezielle Kenntnisse} \"uber unsere Eingabedaten
\begin{itemize}
    \item Das Alter von Kunden in Jahren ist in $[0, 122]$
    \item Die K\"orpergr\"o{\ss}e von Patienten in Zentimetern ist in $[6, 272]$
    \item \emph{Hierarchische Strings}
    \begin{itemize}
        \item Raumnummern an der HU wie etwa \texttt{RUD 25, 4'112}
        \item Standorte in Bibliotheken wie etwa \texttt{ST 270 A149}
    \end{itemize}
    \item UNIX Nutzerrechte wie \texttt{666}
\end{itemize}
Wir k\"onnen hier nicht nur vergleichen, sondern auch die \alert{Struktur} der Datenpunkte ausnutzen.
\end{frame}

\begin{frame}{\"Uberblick \"uber spezielle Sortierverfahren}
In der Vorlesung wurden folgende \emph{spezielle} Sortierverfahren vorgestellt:
\begin{itemize}
    \item Radix Exchange Sort
    \item Bucket Sort
\end{itemize}
Im Folgenden f\"uhren wir \emph{Schreibtischtests} o.g. Verfahren durch.

\medskip

\textbf{Algorithmen folgen!}
\end{frame}
