\section{Amortisierte Analyse}

\begin{frame}{Amortisation in der Wirtschaft}
\resizebox{!}{.8\textheight}{%
\begin{tikzpicture}
\begin{axis}[
    tikzDefaults,
    smooth,
	area style,
    xmin=1,
    xmax=19,
    ticks=none,
    axis x line=middle,
    set layers={background, main}
]
    \addplot+[draw=mandarin, fill=mandarin, fill opacity=0.2] coordinates {
        (1, 289)
        (2, 330)
        (3, 361)
        (4, 393)
        (5, 425)
        (6, 462)
        (7, 62)
        (8, 80)
        (9, 107)
        (10, 141)
        (11, 177)
        (12, 226)
        (13, 279)
        (14, 345)
        (15, 422)
        (16, 512)
        (17, 607)
        (18, 718)
    }
    \closedcycle;
    \addplot+[draw=teal, fill=teal, fill opacity=0.2] coordinates {
        (1, 128)
        (2, 129)
        (3, 126)
        (4, 127)
        (5, 122)
        (6, 127)
        (7, 135)
        (8, 143)
        (9, 148)
        (10, 151)
        (11, 165)
        (12, 169)
        (13, 174)
        (14, 182)
        (15, 196)
        (16, 204)
        (17, 219)
        (18, 222)
    }
    \closedcycle;
    \addplot+[draw=madderlake, fill=madderlake, fill opacity=0.2] coordinates {
        (1, -89)
        (2, -88)
        (3, -95)
        (4, -95)
        (5, -90)
        (6, -90)
        (7, -105)
        (8, -105)
        (9, -104)
        (10, -106)
        (11, -108)
        (12, -109)
        (13, -102)
        (14, -100)
        (15, -102)
        (16, -106)
        (17, -106)
        (18, -105)
    }
    \closedcycle;
\end{axis}
\end{tikzpicture}
}
\end{frame}

\begin{frame}{Selbstorganisierende Listen}

\end{frame}