\section{Suchverfahren}

\begin{frame}{Grundlagen}
\begin{block}{Problemstellung}
Gegeben eine \alert{sortierte} Liste $L = \alglist{a_1, \dots, a_n}$ \"uber einem Wertebereich $D$ und ein beliebiges Element $a \in D$.

Wir wollen die Frage beantworten, ob ein $i \in \{ 1, \dots, n \}$ mit $a_i = a$ existiert.
\end{block}
\begin{block}{Ansatz}<2->
Wir sprechen \"uber \alert{sortierte} Listen, k\"onnen also die \alert{Transitivit\"at} der $\leq_D$-Relation nutzen.

Das bedeutet, dass wir nach Vergleich eines Listenelements $a_i \neq a$ nur noch die Teilliste $L[1, i-1]$ bzw. $L[i+1, n]$ betrachten m\"ussen:
$$L = \alglist{\underbrace{a_1, \dots, a_{i-1}}_{\leq a_i}, a_i, \underbrace{a_{i+1}, \dots, a_n}_{\geq a_i}}$$
\end{block}
\end{frame}

\begin{frame}{Suchverfahren}
In der Vorlesung wurden folgende Suchverfahren vorgestellt:
\begin{itemize}
    \item Binary Search
    \item Fibonacci Search
    \item Interpolation Search
\end{itemize}
Die Verfahren unterscheiden sich im Wesentlichen durch das \alert{Kriterium}, nach dem das n\"achste zu vergleichende Element der Liste gew\"ahlt wird.
\begin{itemize}
    \item Ansonsten wird stets wie zuvor beschrieben die Transitivit\"at ausgenutzt
\end{itemize}

Im Folgenden führen wir \emph{Schreibtischtests} o.g. Verfahren durch.
\end{frame}

\begin{frame}{Binary Search}
\begin{columns}[T,onlytextwidth]
\begin{column}{0.42\textwidth}
\begin{algorithm}[H]
	\caption{Binary Search}
	\label{search:alg:binsearch}
	\DontPrintSemicolon
	\Input{Sortierte Liste $L$ \"uber $D$, \\ Element $a \in D$}
    \Output{Ist $a \in L$?}
    $\ell \gets 1, \quad r \gets n$\;
    \While{$\ell \leq r$}{
        $m \gets \ell + \floor{\frac{r-\ell}{2}}$\;
        \lIf{$a < L_m$}{$r \gets m - 1$}
        \lElseIf{$a > L_m$}{$\ell \gets m + 1$}
        \lElse{\Return{\True}}
    }
    \Return{\False}
\end{algorithm}
\end{column}
\begin{column}{0.53\textwidth}
\begin{task}<2->
Sei $L = \alglist{4, 4, 5, 17, 46, 58, 63}$.
Suchen Sie folgende Elemente in $L$ mittels des nebenstehenden Algorithmus:
\begin{tasks}
    \task $46$
    \task $3$
\end{tasks}
Geben Sie f\"ur jeden Durchlauf die Werte von $\ell$, $r$ und $m$ an!
\end{task}
\end{column}
\end{columns}
\end{frame}

\begin{frame}{Interpolation Search}
\begin{columns}[T,onlytextwidth]
\begin{column}{0.45\textwidth}
\begin{algorithm}[H]
	\caption{Interpolation Search}
	\label{search:alg:intsearch}
	\DontPrintSemicolon
	\Input{Sortierte Liste $L$ \"uber $D$, \\ Element $a \in D$}
    \Output{Ist $a \in L$?}
    $\ell \gets 1, \quad r \gets n$\;
    \While{$\ell \leq r$}{
        \only<1-2>{$m \gets \ell + \floor{(r - \ell) \cdot \red<2->{\frac{a - L_{\ell}}{L_r - L_{\ell}}}}$\;}
        \only<3->{
        \lIf{$L_r = L_{\ell}$}{$m \gets \ell$}
        \lElse{$m \gets \ell + \floor{(r - \ell) \cdot \frac{a - L_{\ell}}{L_r - L_{\ell}}}$}
        \BlankLine
        }
        \lIf{$a < L_m$}{$r \gets m - 1$}
        \lElseIf{$a > L_m$}{$\ell \gets m + 1$}
        \lElse{\Return{\True}}
    }
    \Return{\False}
\end{algorithm}
\end{column}
\begin{column}{0.5\textwidth}
\begin{task}<3->
Sei $L = \alglist{4, 4, 5, 17, 46, 58, 63}$.
Suchen Sie folgende Elemente in $L$ mittels des nebenstehenden Algorithmus:
\begin{tasks}
    \task $46$
    \task $3$
\end{tasks}
Geben Sie f\"ur jeden Durchlauf die Werte von $\ell$, $r$ und $m$ an!
\end{task}
\end{column}
\end{columns}
\end{frame}

\begin{frame}{Fibonacci-Zahlen}
\begin{definition}[Fibonacci-Zahlen]
Die Folge der \alert{Fibonacci-Zahlen} $(F_k)_{k \in \natpos}$ ist induktiv wie folgt definiert:
$$F_1 = 1, \quad F_2 = 1, \quad F_k = F_{k-2} + F_{k-1} \text{ f\"ur alle } k > 2.$$
\end{definition}

\begin{remark}
In den \"Ubungsaufgaben wird zudem gezeigt, dass $F_k \geq \nicefrac{1}{3} \cdot F_{k+2}$ f\"ur alle $k \geq 1$ gilt.
\end{remark}
\end{frame}

\begin{frame}{Fibonacci Search}
\scriptsize
\begin{algorithm}[H]
	\caption{Fibonacci Search}
	\label{search:alg:fibsearch}
	\DontPrintSemicolon
	\Input{Sortierte Liste $L$ \"uber $D$, Element $a \in D$}
    \Output{Ist $a \in L$?}
    $i \gets \min \{ k \in \natpos : F_k > n \}$\;
    $fib_3 \gets F_{i-3}, \quad fib_2 \gets F_{i-2}, \quad m \gets fib_2$\;
    \While{\True}{
        \If{$a < L_m$}{
            \lIf{$fib_2 = 1$}{\Return{\False}}
            \Else{
                $m \gets m - fib_3$\;
                $fib_2 \gets fib_2 - fib_3, \quad fib_3 \gets fib_3 - fib_2$\:
            }
        }
        \ElseIf{$a > L_m$}{
            \lIf{$fib_3 = 0$}{\Return{\False}}
            \Else{
                $m \gets m + fib_3$\;
                $t \gets fib_3, \quad fib_3 \gets fib_2 - fib_3, \quad fib_2 \gets t$\;
            }
        }
        \lElse{\Return{\True}}
    }
    \Return{\False}
\end{algorithm}
\end{frame}
