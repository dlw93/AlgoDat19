\section{\uppercase{$\bigO$}-Notation}

\begin{frame}{Definition}
\begin{columns}[T,onlytextwidth]
\begin{column}{0.47\textwidth}
    \begin{definition}[$\bigO$-Notation]
    Seien $\teal{f}, \red{g}: \nat \to \realpos$ Funktionen.
    Wir schreiben $\teal{f} \in \bigO(\red{g})$ genau dann, wenn es eine Konstante $\orange{c} \in \real_{> 0}$ gibt, sodass ab einem Index $\purple{n_0}$ f\"ur alle $n \geq \purple{n_0}$ gilt, dass $\teal{f}(n) \leq \orange{c} \cdot \red{g}(n)$.
    \end{definition}
    
    \medskip
    
    {\footnotesize Das hei{\ss}t, ab einem gewissen Punkt $\purple{n_0}$ gibt es keinen \alert{wesentlichen} Unterschied mehr zwischen den Funktionswerten von $\teal{f}$ und $\red{g}$.}
\end{column}
\begin{column}{0.47\textwidth}
    \begin{tikzpicture}
        \begin{axis}[
            tikzDefaults, 
            ymax=280,
            xticklabels=\empty,
            yticklabels=\empty,
            legend columns=3,
            extra x ticks={6},
            extra x tick labels={$n_0$},
            extra x tick style={x tick label style={amaranth}, x tick style={amaranth}},
            set layers={background, main},
            clip mode=individual
        ]
            \addplot+[name path=f, domain=-4:12, draw=teal]{2*x^2 - 5*x + 30};
            \addlegendentry{$\teal{f}$}
            \addplot+[domain=-4:12, draw=madderlake]{x^2};
            \addlegendentry{$\red{g}$}
            \addplot+[draw=amaranth, dashed, line width=0.5pt, forget plot] coordinates {(6, 0) (6, 350)};
            \foreach \k in {1, ..., 8}{
                \addplot+[domain=-4:12, draw=mandarin!40, line width=0.5pt, forget plot, on layer=background]{1/2 * \k * x^2};
            }
            \addplot+[name path=gc, domain=-4:12, draw=mandarin, line width=0.6pt, on layer=background]{2 * x^2};
            \addlegendentry{$\orange{c} \cdot \red{g}$}
            %\addplot+[amaranth!17] fill between[of=f and gc, soft clip={domain=6:12}];
        \end{axis}
    \end{tikzpicture}
\end{column}
\end{columns}
\end{frame}

\begin{frame}{Beispiel}
\begin{example}
Sei \emphteal{$f(n) \coloneqq 3 \, n + 5$} und \emphred{$g(n) \coloneqq 2 \, n$}.

Um zu zeigen, dass $\teal{f} \in \bigO(\red{g})$ ist, w\"ahlen wir {$\orange{c=2}$} und erhalten
\begin{equation*}\label{complexity:ex:bigO1}
    \teal{\only<1>{3 \, n} \only<2->{\cancel{3 \, n}} + 5} \leq \orange{2} \cdot \red{2 \, n} = \only<1>{3 \, n} \only<2->{\cancel{3 \, n}} + n.
\end{equation*}
\only<2->{F\"ur $n \geq 5$ ist also $\teal{f(n)} \leq \orange{2} \cdot \red{g(n)}$ und somit $\purple{n_0 = 5}$.}
\end{example}

\begin{remark}<3->
In der Praxis schreibt man h\"aufig $f(n) \in \bigO(g(n))$ anstelle von $f \in \bigO(g)$.

Beispielsweise ist der Ausdruck $\teal{3 \, n + 5} \in \bigO(\red{2 \, n})$ \"aquivalent zu $$\teal{f} \in \bigO(\red{g}) \text{ f\"ur } \teal{f(n) \coloneqq 3 \, n + 5} \text{ und } \red{g(n) \coloneqq 2 \, n}.$$
\end{remark}
\end{frame}

\begin{frame}{Leichte Aufgaben}
\begin{task}<1->
Zeigen Sie, dass $6 \, n + 5 \in \bigO(n)$ ist.
\end{task}
\begin{solution}<2->
F\"ur $c = 7$ erhalten wir $6 \, n + 5 \leq 7 \, n = 6 \, n + n \iff n \geq 5$, also $n_0 = 5$.
\end{solution}

\begin{task}<3->
Zeigen Sie, dass gilt:
\begin{tasks}(3)
    \task $12 \, n - 4 \in \bigO(n)$
    \task $7 \, n + 1 \in \bigO(n^2)$
    \task $\log n^2 \in \bigO(\log n)$ 
\end{tasks}
\end{task}
\begin{solution}<4->
\begin{tasks}(3)
    \task Setze $c=12$ und $n_0=1$
    \task $7 \, n + 1 \leq n^2 \iff n \geq 8$, also $c=1$ und $n_0=8$
    \task Wegen $\log n^2 = 2 \, \log n$ setze $c=2$ und $n_0=1$
\end{tasks}
\end{solution}
\end{frame}

\begin{frame}{Wiederholung: Logarithmusgesetze}
\begin{theorem}
Seien $a, b \in \nat$ und sei $\log \coloneqq \log_x$ f\"ur ein beliebiges, aber festes $x \in \real_{>1}$ sowie $\ln \coloneqq \log_e$. Dann gilt:
\begin{align*}
    \log_b a &= \frac{\log_x a}{\log_x b}, \quad \text{ f\"ur alle $x \in \nat$}\tag{Basiswechsel} \\
    \log(a \cdot b) &= \log a + \log b\tag{Produktregel} \\
    \log(\nicefrac{a}{b}) &= \log a - \log b\tag{Quotientenregel} \\
    \log a^b &= b \, \log a\tag{Potenzregel}\\
    \derivative{n} \ln n &= \nicefrac{1}{n}\tag{Ableitungsregel}
\end{align*}
Falls nicht anders angegeben, nehmen wir zumeist $\log = \log_2$ an.
\end{theorem}
\end{frame}

\begin{frame}{Alternative Definition}
\begin{definition}<+->[$\bigO$-Notation]
Seien $\teal{f}, \red{g}: \nat \to \realpos$ Funktionen.
Wir schreiben $\teal{f} \in \bigO(\red{g})$ genau dann\footnote[frame]{Falls der Grenzwert \emph{nicht} existiert, nutzt man stattdessen den \alert{Limes Superior}: $\limsup_{n \to \infty} \frac{\teal{f(n)}}{\red{g(n)}} < \infty$.}, wenn $\red{g(n)} > 0$ f\"ur alle $n \in \nat$ und $\lim_{n \to \infty} \frac{\teal{f(n)}}{\red{g(n)}} < \infty$.
\end{definition}

\begin{example}<+->
F\"ur $f(n) \coloneqq 3 \, n + 5$ und $g(n) \coloneqq 2 \, n$ erhalten wir $$\lim_{n \to \infty} \frac{f(n)}{g(n)} = \lim_{n \to \infty} \frac{3 \, n + 5}{2 \, n} = \lim_{n \to \infty} \frac{3}{2} + \lim_{n \to \infty} \frac{5}{2 \, n} = \frac{3}{2} + 0 = \frac{3}{2} < \infty.$$
Folglich ist $3 \, n + 5 \in \bigO(2 \, n)$ --- was wir bereits aus Beispiel \ref{complexity:ex:bigO1} wissen.
\end{example}
\end{frame}

\begin{frame}<1>[label=hopital]{Der Satz von L'H\^opital}
\begin{theorem}[Satz von L'H\^opital]
F\"ur Funktionen $f, g : \nat \to \realpos$ mit $\lim_{n \to \infty} f(n) = \lim_{n \to \infty} g(n) \in \{ 0, \infty \}$ gilt: 
\begin{equation*}
    \lim_{n \to \infty} \frac{f(n)}{g(n)} = \lim_{n \to \infty} \frac{f'(n)}{g'(n)}.\only<3->{\tag{\red{*}}}
\end{equation*}
\end{theorem}

\pause

\begin{example}<3->
Sei $f(n) \coloneqq \ln n$ und $g(n) \coloneqq \sqrt{n}$.
Offensichtlich ist $\lim_{n \to \infty} \ln n = \lim_{n \to \infty} \sqrt{n} = \infty$.

Wir k\"onnen also den Satz von L'H\^opital anwenden und erhalten:
$$\lim_{n \to \infty} \frac{\ln n}{\sqrt{n}} \overset{(\red{\star})}{=} \lim_{n \to \infty} \frac{\frac{1}{n}}{\frac{1}{2 \, \sqrt{n}}} = \lim_{n \to \infty} \frac{2 \, \sqrt{n}}{n} = \lim_{n \to \infty} \frac{2}{\sqrt{n}} = 0 < \infty$$
Folglich ist $\ln n \in \bigO(\sqrt{n})$.
\end{example}
\end{frame}

\begin{frame}{Wiederholung: Ableitungsregeln I}
\begin{theorem}\label{complexity:thm:diff_rules}
Seien $f, g: \nat \to \realpos$ Funktionen und sei $(f \cdot g)(n) \coloneqq f(n) \cdot g(n)$.
Dann gilt:
\begin{align*}
    (f \cdot g)' &= f' \cdot g + f \cdot g' \tag{Produktregel} \\
    \left(\frac{f}{g}\right)' &= \frac{f' \cdot g - f \cdot g'}{g^2} \tag{Quotientenregel} \\
    (f \circ g)' &= (f' \circ g) \cdot g' \tag{Kettenregel}
\end{align*}
\end{theorem}

\begin{remark}
Die Quotientenregel folgt direkt aus der Produktregel und der Kettenregel: $$\left(\frac{f}{g}\right)' = (f \cdot g^{-1})' = f' \cdot g^{-1} + f \cdot (-1) \cdot g^{-2} \cdot g' = f' \cdot g \cdot g^{-2} - f \cdot g' \cdot g^{-2} = \frac{f' \cdot g - f \cdot g'}{g^2}$$
\end{remark}
\end{frame}

\begin{frame}{Wiederholung: Ableitungsregeln II}
\begin{task}
Sei $f_b(n) \coloneqq b^n$.
Zeigen Sie, dass f\"ur alle $b \in \natpos$ gilt: $$f_b'(n) = b^n \cdot \ln b$$
\end{task}

\begin{solution}<2->
Wir formen zun\"achst $f_b$ um und erhalten so: $$f_b(n) = \red{b^n = e^{\ln b^n}} = e^{n \, \ln b}.$$
Nun k\"onnen wir unter Verwendung der Kettenregel leicht $f_b'$ bestimmen:
$$f_b'(n) = \left(e^{n \, \ln b}\right)' = e^{n \, \ln b} \cdot \ln b = b^n \cdot \ln b$$
\end{solution}
\end{frame}

\begin{frame}{Wiederholung: Ableitungsregeln III}
\begin{task}
Leiten Sie folgende Funktionen ab, wobei $c \in \realpos$ konstant sei:
\begin{tasks}(3)
    \task $f(n) \coloneqq n^c \cdot \ln n$
    \task $f(n) \coloneqq \ln \ln n$
    \task\label{complexity:task:diff_nn} $f(n) \coloneqq n^n$
\end{tasks}
\end{task}

\begin{solution}<2->
\begin{tasks}
    \task $f'(n) = c \, n^{c-1} \cdot \ln n + n^c \cdot \nicefrac{1}{n} = n^{c-1} \, (c \cdot \ln n + 1)$
    \task $f'(n) = \frac{1}{\ln n} \cdot \nicefrac{1}{n} = \frac{1}{n \, \ln n}$
    \task Wir formen zun\"achst $f$ um und erhalten $f(n) = n^n = e^{\ln n^n} = e^{n \, \ln n}$.
    
    Nun bestimmen wir die Ableitung von $f$: 
    $$f'(n) = \left(e^{n \, \ln n}\right)' = \left(e^{n \, \ln n}\right) \cdot \left(1 \cdot \ln n + n \cdot \nicefrac{1}{n}\right) = n^n \cdot (\ln n + 1)$$
\end{tasks}
\end{solution}
\end{frame}

\againframe<2->{hopital}

\begin{frame}{Vermischte Aufgaben zur \MakeUppercase{$\bigO$}-Notation}
\begin{task}\label{complexity:task:various}
Ordnen Sie folgende Funktionen aufsteigend nach ihrem asymptotischen Wachstum und zeigen Sie die Korrektheit ihrer Anordnung: $$n!, \; 2^{\log n^3}, \; 2^n, \; \log(n^2 + 2^n), \; 2^{3 \, \log \frac{n}{2}}, \; n^n, \; \sqrt{n}$$
\end{task}

\begin{solution}<2->
Beobachte zun\"achst, dass $2^{\log n^3} = n^3$ und $2^{3 \, \log \frac{n}{2}} = 2^{\log \left( \left(\frac{n}{2}\right)^3\right)} = \left(\frac{n}{2}\right)^3 = \nicefrac{1}{8} \cdot n^3$.

Nun ist es leicht zu sehen, dass folgende Anordnung der Aufgabenstellung gen\"ugt: $$\sqrt{n}, \; \log(n^2 + 2^n), \; \underbrace{2^{3 \, \log \frac{n}{2}}}_{=\nicefrac{1}{8} \cdot n^3}, \; \underbrace{2^{\log n^3}}_{=n^3}, \; 2^n, \; n!, \; n^n$$
\end{solution}
\end{frame}

\begin{frame}{Details zur L\"osung}
\begin{itemize}
    \item Um $\sqrt{n} \in \bigO(\log(n^2 + 2^n))$ zu zeigen, sch\"atzen wir wie folgt ab: $$\sqrt{n} \leq n = \log 2^n \leq \log(n^2 + 2^n)$$
    Wir setzen also $c=n_0=1$ und sind fertig.
    \item Bzgl. $\log(n^2 + 2^n) \in \bigO(2^{3 \, \log \frac{n}{2}})$ setzen wir $c=8$ und $n_0=4$ und erhalten: $$\log(n^2 + 2^n) \leq \log(2^n + 2^n) = \log 2^{n+1} = n+1 \leq 8 \cdot \nicefrac{1}{8} \cdot n^3 = c \cdot 2^{3 \, \log \frac{n}{2}}$$
    \item $\nicefrac{1}{8} \cdot n^3 \in \bigO(n^3)$ gilt offensichtlich, da $\nicefrac{1}{8} \cdot n^3 \leq c \cdot n^3$ f\"ur alle $c \geq \nicefrac{1}{8}$ und $n \geq 1$.
    \item Wegen $\lim_{n \to \infty} \frac{n^3}{2^n} = \lim_{n \to \infty} \frac{3 \, n^2}{2^n \cdot \ln 2} = \lim_{n \to \infty} \frac{6 \, n}{2^n \cdot \ln^2 2} = \lim_{n \to \infty} \frac{6}{2^n \cdot \ln^3 2} = 0$ ist $n^3 \in \bigO(2^n)$
    \item F\"ur $2^n \in \bigO(n!)$ w\"ahlen wir $c=2$ und erhalten f\"ur alle $n \geq n_0 = 1$: $$2^n = \underbrace{2 \cdot 2 \cdot \cdots \cdot 2}_{n\text{ Terme}} \leq \underbrace{2 \cdot 2 \cdot 3 \cdot 4 \cdot \cdots \cdot n}_{n \text{ Terme}} = 2 \, n! = c \cdot n!$$
    \item Per Induktion nach $n$ folgt leicht, dass $n! \leq c \cdot n^n$ f\"ur beliebige $c \geq 1$ und alle $n \geq 1$
\end{itemize}
\end{frame}
