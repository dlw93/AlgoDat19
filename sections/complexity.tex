\section{\uppercase{$\bigO$}-Notation}

\begin{frame}{Definition}
\begin{columns}[T,onlytextwidth]
\begin{column}{0.47\textwidth}
    \begin{definition}[$\bigO$-Notation]
    Seien $\teal{f}, \red{g}: \nat \to \realpos$ Funktionen.
    Wir schreiben $\teal{f} \in \bigO(\red{g})$ genau dann, wenn es eine Konstante $\orange{c} \in \real_{> 0}$ gibt, sodass ab einem Index $n_0$ f\"ur alle $n \geq n_0$ gilt, dass $\teal{f}(n) \leq \orange{c} \cdot \red{g}(n)$.
    \end{definition}
    
    \medskip
    
    {\footnotesize Das hei{\ss}t, ab einem gewissen Punkt $n_0$ gibt es keinen \alert{wesentlichen} Unterschied mehr zwischen den Funktionswerten von $\teal{f}$ und $\red{g}$.}
\end{column}
\begin{column}{0.47\textwidth}
    \begin{tikzpicture}
        \begin{axis}[tikzDefaults, xmin=-4, xmax=12, ymin=0, ymax=500]
            \foreach \k in {1, ..., 8}{
                \addplot+[domain=-4:12, draw=mandarin!40, line width=0.5pt, forget plot]{1/2 * \k * x^2};
            }
            \addplot+[domain=-4:12, draw=teal]{2*x^2 - 7*x + 18};
            \addlegendentry{$2 \, x^2 - 7 \, x + 18$}
            \addplot+[domain=-4:12, draw=madderlake]{x^2};
            \addlegendentry{$x^2$}
        \end{axis}
    \end{tikzpicture}
\end{column}
\end{columns}
\end{frame}

\begin{frame}{Beispiel}
\begin{example}
Sei \teal{$f(n) \coloneqq 3 \, n + 5$} und \red{$g(n) \coloneqq 2 \, n$}.

Um zu zeigen, dass $\teal{f} \in \bigO(\red{g})$ ist, w\"ahlen wir \orange{$c=2$} und erhalten
\begin{equation*}
    \teal{\only<1>{3 \, n} \only<2->{\cancel{3 \, n}} + 5} \leq \orange{2} \cdot \red{2 \, n} = \only<1>{3 \, n} \only<2->{\cancel{3 \, n}} + n.
\end{equation*}
\only<2->{F\"ur $n \geq 5$ ist also $\teal{f(n)} \leq \orange{2} \cdot \red{g(n)}$ und somit $n_0 = 5$.}
\end{example}
\end{frame}

\begin{frame}{Bemerkungen}
\begin{remark}<+->
In der Praxis schreibt man h\"aufig $f(n) \in \bigO(g(n))$ anstelle von $f \in \bigO(g)$.

Beispielsweise ist $$\teal{2 \, n^2 + n} \in \bigO(\red{n^2})$$ \"aquivalent zu $$\teal{f} \in \bigO(\red{g}) \text{ f\"ur } \teal{f(n) \coloneqq 2 \, n^2 + n} \text{ und } \red{g(n) \coloneqq n^2}.$$
\end{remark}

\begin{remark}<+->
F\"ur Algorithmus \ref{alg:max_prod} konnten wir zeigen, dass dieser h\"ochstens $4 \cdot \frac{n \, (n-1)}{2} + 2 \in \bigO(n^2)$ Instruktionen ausf\"uhrt.
Demgegen\"uber ben\"otigt Algorithmus \ref{alg:max_prod_ii} nur h\"ochstens $3 \, n \in \bigO(n)$ Instruktionen und ist somit \alert{asymptotisch} effizienter als Algorithmus \ref{alg:max_prod}.
\end{remark}
\end{frame}

\begin{frame}{Eine leichte Aufgabe}
\begin{task}<1->
Sei $f(n) \coloneqq 6 \, n + 7$.

Zeigen Sie, dass $f(n) \in \bigO(n)$ ist, d.h. finden Sie $c \in \real_{> 0}, n_0 \in \natpos$, sodass f\"ur alle $n \geq n_0$ gilt, dass $f(n) \leq c \cdot n$.
\end{task}

\begin{solution}<2->
F\"ur $c = 7$ ist $6 \, n + 7 \leq 7 \, n = 6 \, n + n$ f\"ur alle $n \geq n_0 \coloneqq 7$.
\end{solution}
\end{frame}

\begin{frame}{Mehr Aufgaben}
\begin{task}
Beweisen oder widerlegen Sie:
\begin{subtasks}
    \item $12 \, n - 4 \in \bigO(n)$
    \item $7 \, n + 1 \in \bigO(n^2)$
    \item $\log(n^2) \in \bigO(\log n)$
\end{subtasks}
\end{task}
\end{frame}

\begin{frame}{Alternative Definition}
\begin{definition}[$\bigO$-Notation]
Seien $\teal{f}, \red{g}: \real \to \real$ Funktionen.
Wir schreiben $\teal{f} \in \bigO(\red{g})$ genau dann, wenn $\limsup_{n \to \infty} \frac{\teal{f(n)}}{\red{g(n)}} < \infty$ und $\red{g(n)} > 0$ f\"ur alle $n \in \nat$.
\end{definition}

Das folgende Theorem liefert ein bei dieser Charakterisierung h\"aufig sehr n\"utzliches Werkzeug.

\begin{theorem}[Satz von L'Hospital]
...
\end{theorem}
\end{frame}

\begin{frame}{Ein Beispiel zum Satz von L'Hospital}
\begin{example}

\end{example}
\end{frame}

\begin{frame}{Aufgaben zum Satz von L'Hospital}
\begin{task}

\end{task}
\end{frame}

\begin{frame}{Aufgabe zur Analyse von Algorithmen}
\begin{task}

\end{task}
\end{frame}
