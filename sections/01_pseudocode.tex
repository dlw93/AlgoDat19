\section{Laufzeitanalyse}

\begin{frame}{Hintergrund}
\begin{block}{Knappheit von Ressourcen}
    \begin{itemize}
        \item Speicherplatz (Register, RAM, HDD/SSD, Tape, ...)
        \item \alert{Rechenleistung} (Taktfrequenz)
    \end{itemize}
\end{block}
\begin{block}{Eine Problemstellung -- viele Lösungsansätze}
    \begin{itemize}
        \item Welchen Ansatz sollen wir nutzen?
        \item Ziel: Klassifiziere Lösungsansätze nach ihrer \alert{Eignung}
        \begin{itemize}
            \item Eignung $\approx$ Ressourcenbedarf
        \end{itemize}
    \end{itemize}
\end{block}
\begin{block}{Berechnungsmodell}
    \begin{itemize}
        \item Wir analysieren \alert{abstrakte Algorithmen} -- folglich st\"utzen wir unsere Analysen auf \alert{abstrakte Maschinen}
    \end{itemize}
\end{block}
\end{frame}

\begin{frame}<1-2>[label=first_example]{Ein einfaches Beispiel}
\begin{example}
Gegeben sei eine Liste natürlicher Zahlen $a_1, \dots, a_n \in \nat$.

Maximiere $x = a_i \cdot a_j$ mit $1 \leq i < j \leq n$.

\begin{quote}
    In Worten: Finde das \alert{gr\"o{\ss}te} Produkt, das sich aus zwei \alert{verschiedenen} Elementen der Liste bilden lässt.
\end{quote}

\begin{block}<2->{M\"ogliche Ans\"atze}
    \begin{enumerate}
        \item Betrachte alle möglichen Paare und berechne deren Produkt
        \pause
        \item<4-> Suche die beiden gr\"o{\ss}ten Elemente $a_i, a_j$ und gib $a_i \cdot a_j$ aus
    \end{enumerate}
\end{block}
\end{example}
\end{frame}

\begin{frame}
\frametitle<1>{Algorithmus}
\frametitle<2>{Algorithmus \& Analyse}
\begin{columns}[T,onlytextwidth]
    \begin{column}{0.43\textwidth}
        \vspace{-5pt}
        \begin{algorithm}[H]
        	\caption{Maximales Produkt 1}
        	\label{alg:max_prod}
        	\DontPrintSemicolon
        	\Input{$a_1, \dots, a_n \in \nat$}
            \Output{$\max \{ a_i \cdot a_j : 1 \leq i < j \leq n \}$}
        	$x \gets 0$\;
        	\For{$i \gets 1$ \KwTo $n - 1$}{
        	    \For{$j \gets i + 1$ \KwTo $n$}{
        	        \If{$a_i \cdot a_j > x$}{
        	            $x \gets a_i \cdot a_j$\;
        	        }
                }
            }
            \Return{$x$}
        \end{algorithm}
    \end{column}
    \begin{column}<2>{0.51\textwidth}
        Die \"au{\ss}ere \textbf{for}-Schleife wird $(n-1)$-mal durchlaufen, die innere \textbf{for}-Schleife $n-1, n-2, \dots, 1$-mal
        \begin{itemize}
            \item Anwendung der \alert{gaußschen Summenformel} liefert $\frac{n \, (n-1)}{2}$
        \end{itemize}
        
        Der Aufwand des \textbf{if}-Statements ist von $n$ \alert{unabh\"angig}
        \begin{itemize}
            \item Multiplikationen, Vergleiche und Zuweisungen ben\"otigen lediglich \alert{konstant} viel Zeit
        \end{itemize}
        
        Insgesamt $\leq 4 \cdot \frac{n \, (n-1)}{2} + 2$ Instruktionen
    \end{column}
\end{columns}
\end{frame}

\againframe<3->{first_example}

\begin{frame}
\frametitle<1>{Algorithmus}
\frametitle<2>{Algorithmus \& Analyse}
\begin{columns}[T,onlytextwidth]
    \begin{column}{0.44\textwidth}
        \vspace{-5pt}
        \begin{algorithm}[H]
        	\caption{Maximales Produkt 2}
        	\label{alg:max_prod_ii}
        	\DontPrintSemicolon
        	\Input{$a_1, \dots, a_n \in \nat$}
            \Output{$\max \{ a_i \cdot a_j : 1 \leq i < j \leq n \}$}
        	$k \gets 0, \quad \ell \gets 0$\;
        	\For{$i \gets 1$ \KwTo $n$}{
        	    \If{$a_i > a_k$}{
        	        $\ell \gets k, \quad k \gets i$\;
    	        }
    	        \ElseIf{$a_i > a_{\ell}$}{
    	            $\ell \gets i$\;
    	        }
            }
            \Return{$a_k \cdot a_{\ell}$}
        \end{algorithm}
    \end{column}
    \begin{column}<2>{0.5\textwidth}
        Die \textbf{for}-Schleife wird genau $n$-mal durchlaufen
        
        \medskip
        
        Innerhalb der \textbf{if}-Statements f\"uhren wir lediglich grundlegende Operationen wie Vergleiche und Zuweisungen durch
        
        \medskip
        
        Insgesamt $\leq 3 \, n + 4$ Instruktionen
    \end{column}
\end{columns}
\end{frame}

\begin{frame}<1-2>[label=second_example]{Ein komplexeres Beispiel}
\begin{example}
Wir nennen eine Sequenz $a_1, \dots, a_n \in \nat$ \alert{partitionierbar}, falls  es $k, \ell \in \{1, \dots, n\}$ gibt, sodass $\sum_{i=k}^{\ell} a_i = \frac{1}{2} \cdot  \sum_{i=1}^{n} a_i$.

Entscheide f\"ur eine Sequenz $a_1, \dots, a_n$ mit $\{ a_1, \dots, a_n \} = \{1, \dots, n\}$, ob diese partitionierbar ist.

\begin{quote}
    In Worten: Entscheide f\"ur eine \alert{Permutation} der nat\"urlichen Zahlen von $1$ bis $n$, ob es eine \alert{zusammenh\"angende Subsequenz} gibt, deren Summe die \alert{H\"alfte} der Summe der Gesamtsequenz ist.
\end{quote}

\begin{block}<2->{M\"ogliche Ans\"atze}
    \begin{enumerate}
        \item Berechne die Summe aller möglichen Subsequenzen \emph{from scratch}
        \pause
        \item<4-> Nutze die Distributivit\"at der Addition, um redundante Berechnungen zu vermeiden
    \end{enumerate}
\end{block}
\end{example}
\end{frame}

\begin{frame}
\frametitle<1->{Algorithmus}
\frametitle<2->{Algorithmus \& Analyse}
\begin{columns}[T,onlytextwidth]
    \begin{column}{0.43\textwidth}
        \vspace{-5pt}
        \begin{algorithm}[H]
        	\caption{Subsequenz 1}
        	\label{alg:subseq_part_1}
        	\DontPrintSemicolon
        	\Input{$A = a_1, \dots, a_n \in \nat$}
            \Output{Ist $A$ partitionierbar?}
        	{\color<2>{madderlake}$s \gets \frac{n \, (n+1)}{4}$\;}
        	\For{$k \gets 1$ \KwTo $n$}{
        	    \For{$\ell \gets k$ \KwTo $n$}{
        	        {\color<3>{madderlake}$s' \gets 0$\;}
        	        \For{$i \gets k$ \KwTo $\ell$}{
        	            {\color<5>{madderlake}$s' \gets s' + a_i$\;}
                    }
                    \If{{\color<3>{madderlake}$s' = s$}}{
                        {\color<4>{madderlake}\Return{\True}}
                    }
                }
            }
            {\color<4>{madderlake}\Return{\False}}
        \end{algorithm}
    \end{column}
    \begin{column}<2->{0.51\textwidth}
        \only<2->{
            Die \emphred<2>{Initialisierung} ben\"otigt 4 Instruktionen
            \begin{itemize}
                \item Eine Addition, eine Multiplikation, eine Division und eine Zuweisung
            \end{itemize}
        }
        
        \only<3->{
            Die \emphred<3>{Zuweisung} in Zeile $4$ und der \emphred<3>{Vergleich} in Zeile $6$ werden h\"ochstens $\frac{n \, (n+1)}{2}$-mal ausgef\"uhrt
        }
        
        \medskip
        
        \only<4->{
            Einmalig wird ein Wert \emphred<4>{zurückgegeben}
        }
        
        \medskip
        
        \only<5->{
            Die \emphred<5>{innere \textbf{for}-Schleife} wird $\frac{n \, (n+1) \, (n+2)}{6}$-mal durchlaufen (Details: n\"achste Folie!)
            \begin{itemize}
                \item Entsprechend wird auch Zeile $6$ so oft ausgef\"uhrt
            \end{itemize}
        }
        
        \only<6->{
            Insgesamt $\leq \nicefrac{1}{3} \, n \, (n+1) \, (n+5) + 5$ Instruktionen
        }
    \end{column}
\end{columns}
\end{frame}

\begin{frame}{Details zur Laufzeitschranke}
    \begin{block}{Beobachtungen}
        \begin{itemize}
            \item<+-> Die innere \textbf{for}-Schleife wird stets $(\ell - k + 1)$-mal durchlaufen
            \begin{itemize}
                \item $\underbrace{\underbrace{1}_{\ell=1}, \underbrace{2}_{\ell=2}, \dots, \underbrace{n}_{\ell=n}}_{k=1}\text{-mal}, \underbrace{1, \dots, (n-1)}_{k=2}\text{-mal}, \dots, \underbrace{1, \dots, (n-n+1)}_{k=n}\text{-mal}$
            \end{itemize}
            \item<+-> Als Summe ergibt sich $$\sum_{i=1}^n \frac{i \, (i+1)}{2} = \frac{1}{2} \, \sum_{i=1}^n i \, (i+1) = \frac{1}{2} \, \left({\color<4>{madderlake} \sum_{i=1}^n i^2} + {\color<3->{mandarin} \sum_{i=1}^n i} \right)$$
            \item<+-> Wir wissen bereits, dass \emphorange{$\sum_{i=1}^n i = \frac{1}{2} \, n \, (n+1)$}
            \item<+-> Wir k\"onnen per vollst\"andiger Induktion zeigen, dass \emphred{$\sum_{i=1}^n i^2 = \frac{1}{6} \, n \, (n + 1) \, (2 \, n + 1)$}
        \end{itemize}
    \end{block}
\end{frame}

\begin{frame}{Details zur Laufzeitschranke (2)}
    \begin{lemma}\label{lem:nsquaressum}
    F\"ur alle $n \in \natpos$ gilt $\sum_{i=1}^n i^2 = \frac{1}{6} \, n \, (n + 1) \, (2 \, n + 1)$.
    \end{lemma}
    
    \begin{proof}
    Per Induktion.
    F\"ur $n=1$ gilt Lemma \ref{lem:nsquaressum} offensichtlich.
    F\"ur $n \leadsto n+1$ erhalten wir:
    \begin{align*}
        \sum_{i=1}^{n+1} i^2 = \sum_{i=1}^{n} i^2 + (n+1)^2 &\overset{!}{=} \frac{1}{6} \, n \, (n + 1) \, (2 \, n + 1) + (n+1)^2 \\
        &= \frac{1}{6} \, (n + 1) \, (n \, (2 \, n + 1) + 6 \, (n+1)) \\
        &= \frac{1}{6} \, (n + 1) \, (2 \, n^2 + 7 \, n + 6) \\
        &= \frac{1}{6} \, (n + 1) \, (n + 2) \, (2 \, n + 3)
    \end{align*}
    \end{proof}
\end{frame}

\begin{frame}{Details zur Laufzeitschranke (3)}
    Insgesamt erhalten wir f\"ur Algorithmus \ref{alg:subseq_part_1} unter Verwendung der \emphorange{gaußschen Summenformel} und \emphred{Lemma \ref{lem:nsquaressum}}:
    \begin{align*}
        \frac{1}{2} \, \left({\color{madderlake}\sum_{i=1}^n i^2} + {\color{mandarin}\sum_{i=1}^n i} \right) &= \frac{1}{2} \, \left({\color{madderlake}\frac{1}{6} \, n \, (n + 1) \, (2 \, n + 1)} + {\color{mandarin}\frac{1}{2} \, n \, (n+1)} \right) \\
        &= \frac{1}{2} \cdot \frac{1}{6} \, \left( n \, (n+1) \, (2\,n+1) + 3 \, n \, (n+1) \right) \\
        &= \frac{1}{2} \cdot \frac{1}{6} \, (2\,n+1+3) \, n \, (n+1) \\
        &= \frac{1}{2} \cdot \frac{1}{6} \cdot 2\, (n+2) \, n \, (n+1) \\
        &= \frac{1}{6} \, n \, (n+1) \, (n+2)
    \end{align*}
\end{frame}

\begin{frame}{Bemerkung zum Algorithmus}
    \begin{remark}
        In Algorithmus \ref{alg:subseq_part_1} summieren wir viele Subsequenzen mehrfach.
        
        Sei etwa $A = 4, 2, 3, 1$ mit $n=|A|=4$ die Eingabe.
        Zu Beginn ist $k=1$ und $\ell=1..4$, es werden also nacheinander die Summen $4$, $4+2$, $4+2+3$ und $4+2+3+1$ bestimmt.
        
        Tats\"achlich kann aber jede der Summen durch \alert{eine einzelne Addition} zum Ergebnis der vorherigen Summe bestimmt werden.
        
        So kann die Laufzeit von $\leq \nicefrac{1}{3} \, n \, (n+1) \, (n+5) + 5$ auf $\leq \nicefrac{1}{2} \, (3 \, n^2 + 5 \, n + 10)$ reduziert werden.
    \end{remark}
\end{frame}

\againframe<3->{second_example}

\begin{frame}{Algorithmus \& Analyse}
\begin{columns}[T,onlytextwidth]
\begin{column}{0.43\textwidth}
\vspace{-5pt}
\begin{algorithm}[H]
	\caption{Subsequenz 2}
	\label{alg:subseq_part_2}
	\DontPrintSemicolon
	\Input{$A = a_1, \dots, a_n \in \nat$}
    \Output{Ist $A$ partitionierbar?}
	$s \gets \frac{n \, (n+1)}{4}$\;
	\For{$k \gets 1$ \KwTo $n$}{
	    $s' \gets 0$\;
	    \For{$\ell \gets k$ \KwTo $n$}{
	        $s' \gets s' + a_i$\;
	        
            \If{$s' = s$}{
                \Return{\True}
            }
        }
    }
    \Return{\False}
\end{algorithm}
\end{column}
\begin{column}{0.51\textwidth}
    Die Initialisierung ben\"otigt 4 Instruktionen
    
    \medskip
    
    Der Rumpf der \"au{\ss}eren \textbf{for}-Schleife, und somit auch die Zuweisung in Zeile 3, wird h\"ochstens $n$-mal ausgef\"uhrt
    
    \medskip
    
    Der Rumpf der inneren \textbf{for}-Schleife, bestehend aus einer Addition, einer Zuweisung und einem Vergleich in den Zeilen 5 und 6, wird h\"ochstens $\frac{n \, (n+1)}{2}$-mal ausgef\"uhrt
    
    \medskip
    
    Einmalig wird ein Wert zur\"uckgegeben
    
    \medskip
    
    Insgesamt $\leq \nicefrac{1}{2} \, (3 \, n^2 + 5 \, n + 10)$ Instruktionen
\end{column}
\end{columns}
\end{frame}

\begin{frame}{Details zur Laufzeitschranke}
    Die angegebene Laufzeitschranke von $\nicefrac{1}{2} \, (3 \, n^2 + 5 \, n + 10)$ erhalten wir wie folgt:
    \begin{align*}
        4 + n + 3 \cdot \frac{n \, (n+1)}{2} + 1 &= \nicefrac{1}{2} \cdot 3 \, n \, (n+1) + n + 5 \\
        &= \nicefrac{1}{2} \cdot (3 \, n^2 + 3\, n) + n + 5 \\
        &= \nicefrac{1}{2} \cdot (3 \, n^2 + 3\, n) + \nicefrac{1}{2} \cdot (2 \, n + 10) \\
        &= \nicefrac{1}{2} \cdot (3 \, n^2 + 5\, n + 10)
    \end{align*}

    \begin{remark}
    Insbesondere haben wir damit gezeigt, dass \algref{alg:subseq_part_2} \alert{asymptotisch effizienter} ist als \algref{alg:subseq_part_1}, da die Laufzeit von \algref{alg:subseq_part_2} in $\bigO(n^2)$, die von  \algref{alg:subseq_part_1} hingegen in $\bigO(n^3)$ liegt (siehe Kapitel \nameref{sec:landau}).
    \end{remark}
\end{frame}
